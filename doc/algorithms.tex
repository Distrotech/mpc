\documentclass [12pt]{article}

\usepackage[a4paper]{geometry}
\usepackage[utf8]{inputenc}
\usepackage[T1]{fontenc}
\usepackage{ae}
\usepackage{amsmath,amssymb}
\usepackage{hyperref}
\usepackage{comment}
\usepackage[notref,notcite]{showkeys}

\newcommand {\corr}[1]{{#1}}
\newcommand {\appro}[1]{\widetilde {#1}}
\newcommand {\mpc}{{\tt mpc}}
\newcommand {\mpfr}{{\tt mpfr}}
\newcommand {\ulp}[1]{#1~ulp}
\newcommand {\Ulp}{{\operatorname {ulp}}}
\DeclareMathOperator{\Exp}{\operatorname {Exp}}
\newcommand {\atantwo}{\operatorname {atan2}}
\newcommand{\error}{\operatorname {error}}
\newcommand{\relerror}{\operatorname {relerror}}
\newcommand{\Norm}{\operatorname {N}}
\newcommand {\round}{\operatorname {\circ}}
\DeclareMathOperator{\pinf}{\bigtriangleup}
\DeclareMathOperator{\minf}{\bigtriangledown}
\DeclareMathOperator{\N}{\mathcal N}
\DeclareMathOperator{\A}{\mathcal A}
\newcommand {\Z}{\mathbb Z}
\newcommand {\Q}{\mathbb Q}
\newcommand {\R}{\mathbb R}
\renewcommand {\epsilon}{\varepsilon}
\renewcommand {\theta}{\vartheta}
\renewcommand {\leq}{\leqslant}
\renewcommand {\geq}{\geqslant}
\newcommand {\AGM}{\operatorname{AGM}}

\newtheorem{theorem}{Theorem}
\newtheorem{lemma}[theorem]{Lemma}
\newtheorem{definition}[theorem]{Definition}
\newtheorem{prop}[theorem]{Proposition}
\newtheorem{conj}[theorem]{Conjecture}
\newenvironment{proof}{\noindent{\bf Proof:}}{{\hspace* {\fill}$\blacksquare$}}

\newcommand {\enumi}[1]{(\alph {#1})}
\renewcommand {\labelenumi}{\enumi {enumi}}
\newcommand {\enumii}[1]{(\roman {#1})}
\renewcommand {\labelenumii}{\enumii {enumii}}

\title {MPC: Algorithms and Error Analysis}
\author {Andreas Enge \and Philippe Th\'eveny \and Paul Zimmermann}
\date {Draft; September 18, 2012}

\begin {document}
\maketitle
\tableofcontents


\section {Error propagation}

\subsection {Introduction and notation}

This section is devoted to the analysis of error propagation: Given a function
whose input arguments already have a certain error, what is the error bound on
the function output? The output error usually consists of two components: the
error propagated from the input, which may be arbitrarily amplified (or, if
one is lucky, shrunk); and an
additional small error accounting for the rounding of the output. The results
are needed to give a cumulated error analysis for algorithms that combine
several elementary arithmetic operations.


\subsubsection {Ulp calculus}

\begin {definition}
\label {def:exp}
Let $x$ be a real number, which can be written uniquely as
$x = m \cdot 2^e$ with $\frac{1}{2} \le |m| < 1$.
The {\em exponent} of $x$ is
$\Exp(x) = e = \lfloor \log_2 |x| \rfloor + 1$.
The number is {\em representable at precision~$p$} if
$2^p m$ is an integer.
We denote the rounding of $x$ to one of the at most two representable
numbers in the open interval $(x - 2^{-p}, x + 2^{-p})$ by
$\round (x) = \round_p (x)$, with rounding being to nearest, up, down,
towards zero or away from zero if there is a choice.
\end {definition}

\begin {prop}
\label {prop:expmuldiv}
If $x_1$ and $x_2$ are two real numbers, then
\begin {gather*}
\Exp (x_1) + \Exp (x_2) - 1 \leq \Exp (x_1 x_2) \leq \Exp (x_1) + \Exp (x_2),
\\
\Exp (x_1) - \Exp (x_2) \leq \Exp \left( \frac {x_1}{x_2} \right)
\leq \Exp (x_1) - \Exp (x_2) + 1.
\end {gather*}
\end {prop}

\begin {proof}
Write $x_n = m_n 2^{\Exp (x_n)}$ and
$x = x_1 x_2 = m 2^{\Exp x} = m_1 m_2 2^{\Exp (x_1) + \Exp (x_2)}$
with $\frac {1}{2} \leq m_n, m < 1$.
Then $m = m_1 m_2$ if the product is at least $\frac {1}{2}$ and
$m = 2 m_1 m_2$ if the product is less than $\frac {1}{2}$, which
yields the first line of inequalities.
The other inequalities are derived in the same way from
$\frac {1}{2} < \frac {m_1}{m_2} < 2$.
\end {proof}


\begin {prop}
\label {prop:expround}
For any real number $x$,
\[
\Exp (x) \leq \Exp (\round (x)) \leq \Exp (x) + 1,
\]
with equality occurring on the right if and only if
$|x|$ has been rounded up to $|\round (x)| = 2^{\Exp (x)}$.
\end {prop}

\begin {proof}
Letting $x = m 2^{\Exp (x)}$, we have
$\frac {1}{2} \cdot 2^{\Exp (x)} \leq \round (x) \leq 1 \cdot 2^{\Exp (x)}$,
since these two numbers are representable (independently of the precision).
\end {proof}


\begin {definition}
\label {def:ulp}
Let $x$ be a real number which is representable at precision~$p$.
Its associated {\em unit in the last place} is
$\Ulp(x) = \Ulp_p (x) = 2^{\Exp(x) - p}$, so that adding $\Ulp(x)$ to $x$
corresponds to adding~$1$ to the integer $2^p m$.
\end {definition}


\subsubsection {Absolute error}

In the remainder of this chapter, all complex numbers are denoted by
the letter $z$ with subscripts and mathematical accents, decomposed in
Cartesian coordinates as $z = x + i y$ with the same diacritics applied
to $x$ and $y$ as to $z$. All representable real numbers are supposed
to have the same precision~$p$. We apply the following error definition
of real numbers separately to the two coordinates of a complex number.

\begin {definition}
\label {def:error}
Given a real number $\corr x$ and its approximation $\appro x$,
we define the {\em absolute error} of $\appro x$ as
$\error (\appro x) = | \corr x - \appro x |$.
\end {definition}

Notice that in the following, the absolute error is usually expressed in terms
of $\Ulp$, which is itself a relative measure with respect to the exponent of
the number.

Let $\corr z = f (\corr {z_1}, \ldots) = \corr x + i \corr y$ be the correct
result of a complex function applied to the correct arguments $\corr {z_n}$.
We assume that the $\corr {z_n}$ themselves are not known, but only
approximate input values $\appro {z_n} = \appro {x_n} + i \appro {y_n}$;
for instance, the $\corr {z_n}$ may be the exact results of some formul\ae,
whereas the $\appro {z_n}$ are the outcome of the corresponding computation
and affected by rounding errors. We suppose that error bounds
$\error (\appro {x_n}) \leq k_{R, n} 2^{\Exp (\appro {x_n}) - p}$
and $\error (\appro {y_n}) \leq k_{I, n} 2^{\Exp (\appro {y_n}) - p}$ for
some $k_{R, n}$ and $k_{I, n}$ are known. (This particular notation
becomes more comprehensible when $\appro {x_n}$ and $\appro {y_n}$ are
representable at precision~$p$, since then the units of the error measure
become $\Ulp (\appro {x_n})$ and $\Ulp (\appro {y_n})$, respectively;
however, there is no need to restrict the results of this chapter to
representable numbers.)
Our aim is to determine the propagated error in the output value
$\appro z = \appro x + i \appro y = f (\appro {z_1}, \ldots)$, which is given by
\begin {equation}
\label {eq:properror}
\error (\appro x)
\leq | \Re (f (\corr {z_1}, \ldots)) - \Re (f (\appro {z_1}, \ldots)) |
\end {equation}
and an analogous formula for $\error (\appro y)$. In general,
we are looking for $k_R$ and $k_I$ such that
\[
\error (\appro x) \leq k_R 2^{\Exp (\appro x) - p}
\text { and }
\error (\appro y) \leq k_I 2^{\Exp (\appro y) - p}.
\]
Moreover, we are interested in the cumulated error if additionally
$\appro z$ is rounded coordinatewise at the target precision~$p$
to $\round (\appro z)$. This operation adds an error of
$c_R \Ulp (\round (\appro x))$ to the real and of
$c_I \Ulp (\round (\appro y))$ to the imaginary part, where
$c_X \leq 1$ when $\round$ stands for rounding up, down, to zero or
to infinity, and $c_X \leq \frac {1}{2}$ when $\round$ stands for
rounding to nearest.
Then, via Proposition~\ref {prop:expround},
\[
\error (\round (\appro x)) \leq (k_R + c_R) \Ulp (\appro x)
\text { and }
\error (\round (\appro y)) \leq (k_I + c_I) \Ulp (\appro y).
\]


\subsubsection {Real relative error}

It can sometimes be useful to determine errors not absolutely as differences
(close to~$0$),
but relatively as multiplicative factors (close to~$1$).

\begin {definition}
\label {def:relerror}
Given a real number $\corr x$ and its approximation $\appro x$,
we define the {\em upper relative error} of $\appro x$ by
\[
\relerror^+ (\appro x) = \epsilon^+
= \frac {\max (0, \corr x - \appro x)}{|\appro x|}.
\]
and its {\em lower relative error} by
\[
\relerror^- (\appro x) = \epsilon^-
= \frac {\max (0, \appro x - \corr x)}{|\appro x|}.
\]
The {\em relative error} of $\appro x$ is
\[
\relerror (\appro x) = \epsilon = \max (\epsilon^-, \epsilon^+)
= \frac {\error (\appro x)}{|\appro x|}.
\]
\end {definition}

Notice that $\epsilon^- = 0$ whenever $\appro x \leq \corr x$ and
$\epsilon^+ = 0$ whenever $\appro x \geq \corr x$, so that at least one of
$\epsilon^+$ and $\epsilon^-$ is zero. We express relative errors with
respect to the approximate value; thus, they define an interval around
the approximate value containing the correct one. Precisely,
$\corr x = \appro x (1 + \vartheta)$ for some number $\theta$ with
$|\theta| \leq \epsilon$.

The definition of relative error carries over to complex numbers,
see \S\ref {sssec:comrelerror}.
However, in the following we usually argument separately for the two coordinates,
so we use corresponding $\epsilon$-values with subscript $R$ and $I$ for the
real and imaginary part, respectively.

When an absolute error is expressed in the relative unit $\Ulp$, then
it is easy to switch back and forth between absolute and relative errors.

\begin {prop}
\label {prop:relerror}
Let $\appro x$ be a real number representable at precision $p$.
\begin {enumerate}
\item
If $\error (\appro x) \leq k \Ulp (\appro x)$,
then $\relerror (\appro x) \leq k 2^{1 - p}$.
\item
If $\relerror (\appro x) \leq k 2^{-p}$,
then $\error (\appro x) \leq k \Ulp (\appro x)$.
\end {enumerate}
These assertions remain valid if $\appro x$ is not representable at
precision~$p$ and $\Ulp (\appro x)$ is replaced by $2^{\Exp (\appro x) - p}$.
\end {prop}

\begin {proof}
Concerning the first assertion, we have
$
\relerror (\appro x) = \frac {\error (\appro x)}{|\appro x|}
\leq
\frac {k \Ulp (\appro x)}{|\appro x|}.
$
Plugging in from Definition~\ref {def:ulp} that
$\Ulp (\appro x) = 2^{\Exp (\appro x) - p}$ and
$|\appro x| \geq 2^{\Exp (\appro x) - 1}$ finishes the proof.
The second assertion is proved in the same manner, using
$|\appro x| \leq 2^{\Exp (\appro x)}$.
\end {proof}


\subsubsection {Complex relative error}
\label {sssec:comrelerror}

The sign of errors is not meaningful any more in the complex case, and
only the notion of total relative error carries over.

\begin {definition}
\label {def:comrelerror}
Given a complex number $\corr z$ and its non-zero approximation
$\appro z$, let
\[
\theta = \frac {\corr z - \appro z}{\appro z},
\text { or }
\corr z = (1 + \theta) \appro z.
\]
Then the {\em relative error} of $\appro z$ is
\[
\relerror (\appro z) = \epsilon = | \theta |
= \left| \frac {\corr z - \appro z}{\appro z} \right|.
\]
\end {definition}

The following result gives a coarse
estimate of the relative errors of the real and imaginary parts in terms of
the complex relative error, and vice versa.

\begin {prop}
\label {prop:comrelerror}
Let $\corr z = \corr x + i \corr y$, $\appro z = \appro x + i \appro y$,
$\epsilon = \relerror (\appro z)$,
$\epsilon_R = \relerror (\appro x)$ and
$\epsilon_I = \relerror (\appro y)$. Then
\begin {align*}
\epsilon_R
&\leq \left( 1 + 2^{\Exp (\appro y) - \Exp (\appro x) + 1} \right) \epsilon, \\
\epsilon_I
&\leq \left( 1 + 2^{\Exp (\appro x) - \Exp (\appro y) + 1} \right) \epsilon, \\
\epsilon
&\leq \max \left( \epsilon_R, \epsilon_I \right).
\end {align*}
\end {prop}

\begin {proof}
Write $\theta = \frac {\corr z - \appro z}{\appro z}
= \theta_R + i \theta_I$. Then
$\corr x - \appro x = \Re (\appro z \theta)
= \appro x \theta_R - \appro y \theta_I$, and
\begin {align*}
\epsilon_R
&= \left| \frac {\corr x - \appro x}{\appro x} \right|
\leq |\theta_R| + \left| \frac {\appro y}{\appro x} \right| |\theta_I|
\leq \left( 1 + \left| \frac {\appro y}{\appro x} \right| \right)
\max (|\theta_R|, |\theta_I|) \\
&\leq \left( 1 + 2^{\Exp (\appro y) - \Exp (\appro x) + 1} \right)
|\theta|
\end {align*}
by Proposition~\ref {prop:expmuldiv}. The second inequality is proved
in the same way.

For the converse direction, write
\[
|\theta|^2
= \frac {(\corr x - \appro x)^2 + (\corr y - \appro y)^2}{\appro x^2 + \appro y^2}
= \frac {\epsilon_R^2 \appro x^2 + \epsilon_I^2 \appro y^2}{\appro x^2 + \appro y^2}
\leq \max \left( \epsilon_R^2, \epsilon_I^2 \right).
\]
\end {proof}

As a corollary to Propositions~\ref {prop:relerror}
and~\ref {prop:comrelerror}, we obtain the following result that shows how
to transform absolute into complex relative errors.

\begin {prop}
\label {prop:comabstorelerror}
Let $\appro z = \appro x + i \appro y$ be representable at precision~$p$.
If
$\error (\appro x) \leq k \, \Ulp (\appro x)$ and
$\error (\appro y) \leq k \, \Ulp (\appro y)$, then
\[
\relerror (\appro z) \leq k \, 2^{1-p}.
\]
As in Proposition~\ref {prop:relerror}, there is an immediate generalisation
if $\appro z$ is not representable at precision~$p$.
\end {prop}

This result can be used to analyse how rounding affects complex
relative errors.

\begin {prop}
\label {prop:comrelround}
Let $\relerror (\appro z) = \epsilon$.
Then
\[
\relerror (\round (\appro z)) \leq
\epsilon + c (1 + \epsilon) 2^{1-p},
\]
where $c = \frac {1}{2}$ if both the real and the imaginary part are
rounded to nearest, and $c = 1$ otherwise.
\end {prop}

\begin {proof}
Write $\corr z = (1 + \theta) \appro z$ with $|\theta| = \epsilon$.
Applying Proposition~\ref {prop:comabstorelerror} with $\appro z$ in the
place of $\corr z$, $\round (\appro z)$ in the place of $\appro z$
and $c$ in the place of~$k$,
there is a $\theta'$ with $\appro z = (1 + \theta') \round (\appro z)$
and $|\theta'| \leq c \, 2^{1-p}$.
So $\corr z = (1 + \theta'') \round (\appro z)$ with
$\theta'' = (1 + \theta)(1 + \theta') - 1
= \theta + \theta' (1 + \theta)$.
\end {proof}


\subsection {Real functions}

In this section, we derive for later use results on error propagation for
functions with real arguments and values. Those already contained in
\cite{MPFRAlgorithms} are simply quoted for the sake of self-containedness.



\subsubsection {Division}
\label {sssec:proprealdiv}

Let
\[
\appro x = \frac {\appro {x_1}}{\appro {x_2}}.
\]
Then
\[
\error (\appro x) = \left|
\frac {\corr {x_1}}{\corr {x_2}} - \frac {\appro {x_1}}{\appro {x_2}} \right|
= \left| \frac {\appro {x_1}}{\appro {x_2}} \right|
\cdot \left|
1 - \left| \frac {\corr {x_1}}{\appro {x_1}} \right|
   \cdot \left| \frac {\appro {x_2}}{\corr {x_2}} \right|
\right|
= | \appro x |
\cdot \left|
1 - \left| \frac {\corr {x_1}}{\appro {x_1}} \right|
   \cdot \left| \frac {\appro {x_2}}{\corr {x_2}} \right|
\right|
\]
Using the notation introduced in Definition~\ref {def:relerror} together
with the obvious subscripts to the $\epsilon$, we obtain the bounds
\[
- \frac {\epsilon_1^+ + \epsilon_2^-}{1 - \epsilon_2^-}
=
1 - \frac {1 + \epsilon_1^+}{1 - \epsilon_2^-}
\leq
1 - \left| \frac {\corr {x_1}}{\appro {x_1}} \right|
   \cdot \left| \frac {\appro {x_2}}{\corr {x_2}} \right|
\leq
1 - \frac {1 - \epsilon_1^-}{1 + \epsilon_2^+}
=
\frac {\epsilon_1^- + \epsilon_2^+}{1 + \epsilon_2^+}
\]
We need to make the assumption that $\epsilon_2^- < 1$, which is reasonable
since otherwise the absolute error on $\appro {x_2}$ would exceed the number
itself. Then
\[
\error (\appro x)
\leq
\max \left(
   \frac {\epsilon_1^+ + \epsilon_2^-}{1 - \epsilon_2^-},
   \frac {\epsilon_1^- + \epsilon_2^+}{1 + \epsilon_2^+}
\right) |\appro x|
\leq
\frac {\epsilon_1 + \epsilon_2}{1 - \epsilon_2^-} 2^{\Exp (\appro x)}
\]
Using the estimation of the relative in terms of the absolute error of
Proposition~\ref {prop:relerror}, this bound can be translated into
\begin {equation}
\label {eq:proprealdiv}
\error (\appro x)
\leq
\frac {2 (k_1 + k_2)}{1 - \epsilon_2^-} 2^{\Exp (\appro x) - p}
\leq
\frac {2 (k_1 + k_2)}{1 - k_2 2^{1 - p}} 2^{\Exp (\appro x) - p}
\end {equation}


\subsubsection {Square root}
\label {sssec:proprealsqrt}

Let
\[
\appro x = \sqrt {\appro {x_1}}.
\]
Then by \cite[\S1.7]{MPFRAlgorithms},
\begin {equation}
\label {eq:proprealsqrt}
\error (\appro x)
\leq
\frac {2 k_1}{1 + \sqrt {1 + \epsilon_1^-}} 2^{\Exp (\appro x) - p}.
\end {equation}


\subsubsection {Cosine and sine}
\label {sssec:proprealcossin}

Let
\[
\appro x = \cos {\appro {x_1}}.
\]
By the mean value theorem, there is a $\xi$ between $x_1$ and $\appro {x_1}$
such that
\[
\cos (x_1) - \cos (\appro {x_1}) = -\sin (\xi) (x_1 - \appro {x_1}),
\]
so that
\[
\error (\appro x)
\leq \error (\appro {x_1}).
\]
Taking the exponents into account, one obtains
\begin {equation}
\label {eq:proprealcos}
\error (\appro x)
\leq
k \, 2^{\Exp (\appro {x_1}) - \Exp (\appro x)}
\, 2^{\Exp (\appro x) - p}.
\end {equation}

For the sine function, a completely analogous argument shows that
\eqref {eq:proprealcos} also holds.


\subsubsection {Logarithm}
\label {sssec:propreallog}

Let
\[
\appro x = \log (1 + \appro {x_1})
\]
for $\appro {x_1} > -1$.
By the mean value theorem, there is a $\xi$ between $x_1$ and $\appro {x_1}$
such that
\[
\error (\appro x) = \frac {1}{1 + \xi} \error (\appro {x_1})
\leq \frac {1}{1 + \min (x_1, \appro {x_1})} \error (\appro {x_1}).
\]
For $x_1 > 0$, this implies
\begin {eqnarray*}
\error (\appro x)
& \leq & \error (\appro {x_1})
\leq
k \, 2^{\Exp (\appro {x_1}) - \Exp (\appro x)}
\, 2^{\Exp (\appro x) - p} \\
& \leq & 2 \, k \, \frac {\appro {x_1}}{\appro x} \, 2^{\Exp (\appro x) - p} \\
& \leq & 2 \, k \, \frac {\appro {x_1}}{\appro {x_1} - \appro {x_1}^2/2}
\, 2^{\Exp (\appro x) - p}
\end {eqnarray*}
using $\log (1 + z) \geq z - z^2/2$ for $z > 0$.
For $0 < x_1 \leq 1$, we have $\appro {x_1}^2/2 \leq \appro {x_1}/2$ and
\[
\error (\appro x)
\leq 4 \, k \, 2^{\Exp (\appro x) - p}.
\]


\subsection {Complex functions}

\subsubsection {Addition/subtraction}

Using the notation introduced above, we consider
\[
\appro z = \appro {z_1} + \appro {z_2}.
\]
By \eqref {eq:properror}, we obtain
\begin{align*}
\error (\appro x)
& \leq | (\corr {x_1} + \corr {x_2}) - (\appro {x_1} + \appro {x_2})|
\\
& \leq | \corr {x_1} - \appro {x_1} | + | \corr {x_2} - \appro {x_2}|
\\
& \leq k_{R,1} 2^{\Exp (\appro {x_1}) - p}
+ k_{R,2} 2^{\Exp (\appro {x_2}) - p}
\\
& \leq \left( k_{R,1} 2^{d_{R,1}} + k_{R,2} 2^{d_{R,2}} \right)
2^{\Exp (\appro x) - p},
\end{align*}
where $d_{R,n}=\Exp(\appro {x_n})-\Exp(\appro x)$.
Otherwise said, the absolute errors add up, but their relative expression
in terms of $\Ulp$ of the result grows if the result has a smaller
exponent than the operands, that is, if cancellation occurs.

If $\appro {x_1}$ and $\appro {x_2}$, have the same sign, then there
is no cancellation, $d_{R, n} \leq 0$ and
\[
\error (\appro x) \leq (k_{R,1} + k_{R,2}) 2^{\Exp (\appro x) - p}.
\]

An analogous error bound holds for the imaginary part.

For subtraction, the same bounds are obtained, except that the simpler bound
now holds whenever $\appro {x_1}$ and $\appro {x_2}$ resp.
$\appro {y_1}$ and $\appro {y_2}$ have different signs.

We obtain a useful result for complex relative errors when $z_1$ and $z_2$
lie in the same quadrant, so that cancellation occurs neither for the real
nor for the imaginary part.

\begin {lemma}
\label {lm:arithgeom}
Let $c_1 = a_1 + i b_1$ and $c_2 = a_2 + i b_2$ lie in the same quadrant,
that is, $a_1 a_2$, $b_1 b_2 \geq 0$. Then
\[
|c_1| + |c_2| \leq \sqrt 2 \cdot |c_1 + c_2|.
\]
\end {lemma}

\begin {proof}
One readily verifies that
\[
2 |c_1 + c_2|^2 - (|c_1| + |c_2|)^2
= (|c_1| - |c_2|)^2 + 4 (a_1 a_2 + b_1 b_2)
\geq 0
\]
\end {proof}

Assume now that $\corr {z_n} = (1 + \theta_n) \appro {z_n}$ for $n=1,2$
with $\epsilon_n = |\theta_n|$ lie in the same quadrant.
Then
$\corr z = (1 + \theta) \appro z$
with
\[
\theta = \frac {\theta_1 \appro {z_1} + \theta_2 \appro {z_2}}
               {\appro {z_1} + \appro {z_2}}.
\]
and
\begin {equation}
\label {eq:propaddrel}
\relerror (\appro z)
\leq
\max (\epsilon_1, \epsilon_2)
   \frac {|\appro {z_1}| + |\appro {z_2}|}{|\appro {z_1} + \appro {z_2}|}
\leq
\sqrt 2 \, \max (\epsilon_1, \epsilon_2)
\end {equation}
by Lemma~\ref {lm:arithgeom}.




\subsubsection {Multiplication}
\label {sssec:propmul}

Let
\[
\appro z = \appro {z_1} \times \appro {z_2},
\]
so that
\begin {align*}
\appro x & = \appro {x_1} \appro {x_2} - \appro {y_1} \appro {y_2}, \\
\appro y & = \appro {x_1} \appro {y_2} + \appro {x_2} \appro {y_1}.
\end {align*}
Then
\[
\error (\appro x)
\leq | \Re (\corr {z_1} \times \corr {z_2})
- \Re (\appro {z_1} \times \appro {z_2})|
\leq
| \corr {x_1} \corr {x_2} - \appro {x_1} \appro {x_2}|
+ | \corr {y_1} \corr {y_2} - \appro {y_1} \appro {y_2}|.
\]
The first term on the right hand side can be bounded as follows,
where we use the short-hand notation $\epsilon_{R, 1}^+$ for
$\relerror^+ (\appro {x_1})$, and analogously for other relative errors:
\begin{align*}
| \corr {x_1} \corr {x_2} - \appro {x_1} \appro {x_2}|
& \leq
\frac{1}{2} \left(
  |\appro {x_1} - \corr {x_1}| (|\appro {x_2}| + |\corr {x_2}|)
+ |\appro {x_2} - \corr {x_2}| (|\appro {x_1}| + |\corr {x_1}|)
\right)
\\
& \leq \frac {1}{2} \left(
  \epsilon_{R, 1} |\appro {x_1}| |\appro {x_2}|
  \left( 1 + \frac {|\corr {x_2}|}{|\appro {x_2}|} \right)
+ \epsilon_{R, 2} |\appro {x_2}| |\appro {x_1}|
  \left( 1 + \frac {|\corr {x_1}|}{|\appro {x_1}|} \right)
  \right)
\\
& \leq \left(
  k_{R, 1}
  \left( 1 + \frac {|\corr {x_2}|}{|\appro {x_2}|} \right)
+ k_{R, 2}
  \left( 1 + \frac {|\corr {x_1}|}{|\appro {x_1}|} \right)
  \right) |\appro {x_1} \appro {x_2}| \, 2^{-p}
  \text { by Proposition~\ref {prop:relerror}}
\\
& \leq \left(
   k_{R, 1} (2 + \epsilon_{R, 2}^+)
   + k_{R, 2} (2 + \epsilon_{R, 1}^+)
   \right) 2^{\Exp (\appro {x_1} \appro {x_2}) - p}.
\end{align*}
In the same way, we obtain
\[
| \corr {y_1} \corr {y_2} - \appro {y_1} \appro {y_2}|
\leq \left(
   k_{I, 1} (2 + \epsilon_{I, 2}^+)
   + k_{I, 2} (2 + \epsilon_{I, 1}^+)
   \right) 2^{\Exp (\appro {y_1} \appro {y_2}) - p}.
\]

It remains to estimate $\Exp (\appro {x_1} \appro {x_2})$ and
$\Exp (\appro {y_1} \appro {y_2})$ with respect to $\Exp (\appro x)$ to obtain
a bound in terms of $\Ulp (\appro x)$. This becomes problematic when, due
to the subtraction, cancellation occurs. In all generality, let
$d = \Exp (\appro {x_1} \appro {x_2}) - \Exp (\appro x)
\leq \Exp (\appro {x_1}) + \Exp (\appro {x_2}) - \Exp (\appro x)$
by Proposition~\ref {prop:expmuldiv} and
$d' = \Exp( \appro {y_1} \appro {y_2}) - \Exp (\appro x)
\leq \Exp (\appro {y_1}) + \Exp (\appro {y_2}) - \Exp (\appro x)$.
Then
\begin {equation}
\label {eq:propmulre}
\error( \appro x) \leq \left(
   \left( k_{R, 1} (2 + \epsilon_{R, 2}^+)
   + k_{R, 2} (2 + \epsilon_{R, 1}^+) \right) 2^d
   + \left( k_{I, 1} (2 + \epsilon_{I, 2}^+)
   + k_{I, 2} (2 + \epsilon_{I, 1}^+) \right) 2^{d'}
   \right) 2^{\Exp (\appro x) - p}.
\end {equation}
If $\appro {x_1} \appro {x_2}$ and $\appro {y_1} \appro {y_2}$ have different
signs, then there is no cancellation, and, using the monotonicity of the
exponent with respect to the absolute value, we obtain
\[
\Exp (\appro x) = \Exp (\appro {x_1} \appro {x_2} - \appro {y_1} \appro {y_2})
= \Exp (|\appro {x_1} \appro {x_2}| + |\appro {y_1} \appro {y_2}|)
\geq \Exp (|\appro {x_1} \appro {x_2}|), \Exp (|\appro {y_1} \appro {y_2}|),
\]
so that $d$, $d' \leq 0$ and the error bound simplifies as
\[
\error( \appro x) \leq \left(
   k_{R, 1} (2 + \epsilon_{R, 2}^+)
   + k_{R, 2} (2 + \epsilon_{R, 1}^+)
   + k_{I, 1} (2 + \epsilon_{I, 2}^+)
   + k_{I, 2} (2 + \epsilon_{I, 1}^+)
   \right) 2^{\Exp (\appro x) - p}.
\]

The same approach yields the error of the imaginary part. Letting
$\delta = \Exp (\appro {x_1} \appro {y_2}) - \Exp (\appro y)
\leq \Exp( \appro {x_1}) + \Exp (\appro {y_2}) - \Exp (\appro y)$ and
$\delta' = \Exp (\appro {x_2} \appro {y_1}) - \Exp (\appro {y})
\leq \Exp (\appro {x_2}) + \Exp (\appro {y_1}) - \Exp (\appro y)$,
it becomes
\begin {equation}
\label {eq:propmulim}
\error( \appro y) \leq \left(
   \left( k_{R, 1} (2 + \epsilon_{I, 2}^+)
   + k_{I, 2} (2 + \epsilon_{R, 1}^+) \right) 2^{\delta}
   + \left( k_{I, 1} (2 + \epsilon_{R, 2}^+)
   + k_{R, 2} (2 + \epsilon_{I, 1}^+) \right) 2^{\delta'}
   \right) 2^{\Exp (\appro y) - p}.
\end {equation}
If $\appro {x_1} \appro {y_2}$ and $\appro {x_2} \appro {y_1}$ have
the same sign, then $\delta$, $\delta' \leq 0$ and
\[
\error( \appro y) \leq \left(
   k_{R, 1} (2 + \epsilon_{I, 2}^+)
   + k_{I, 2} (2 + \epsilon_{R, 1}^+)
   + k_{I, 1} (2 + \epsilon_{R, 2}^+)
   + k_{R, 2} (2 + \epsilon_{I, 1}^+)
   \right) 2^{\Exp (\appro y) - p}.
\]

Notice that $x_1 x_2$ and $y_1 y_2$ have the same sign
if and only if $y_1 x_2$ and $x_1 y_2$ do. So there is
always cancellation in precisely one of the real or the imaginary part.

The different values $\epsilon_{X, n}^+$ for $X \in \{ R, I \}$ and
$n \in \{ 1, 2 \}$ in the formul{\ae} above may be bounded by
$k_{X, n} 2^{1 - p}$ according to Proposition~\ref {prop:relerror}.
If some $|\appro {x_n}| \geq |\corr {x_n}|$ resp.
$|\appro {y_n}| \geq |\corr {y_n}|$ (for instance, because they have been
computed by rounding away from zero), then the corresponding
$\epsilon_{X, n}^+$ are zero.


A coarser bound may be obtained more easily by considering complex
relative errors. Write $\corr {z_n} = (1 + \theta_n) \appro {z_n}$
with $\epsilon_n = | \theta_n |$. Then $\corr z = (1 + \theta) \appro z$
with $\theta = \theta_1 + \theta_2 + \theta_1 \theta_2$ and
\begin {equation}
\label {eq:propmulrel}
\epsilon = \relerror (\appro z)
\leq \epsilon_1 + \epsilon_2 + \epsilon_1 \epsilon_2.
\end {equation}
By Proposition~\ref {prop:relerror},
we have $\epsilon_{X, n} \leq k_{X, n} 2^{1-p}$ for $X \in \{ R, I \}$,
and by Proposition~\ref {prop:comrelerror},
$\epsilon_n \leq  \max (k_{R, n}, k_{I, n}) 2^{1 - p}$.
Under normal circumstances, $\epsilon_1 \epsilon_2$ should be negligible;
for instance, if we assume that
$\max (k_{R, 1}, k_{I, 1}) \max (k_{R, 2}, k_{I, 2}) \leq 2^{p - 1}$,
then $\epsilon_1 \epsilon_2 \leq 2^{1-p}$ and
$\epsilon \leq \big( \max (k_{R, 1}, k_{I, 1}) + \max (k_{R, 2}, k_{I, 2}) + 1
\big) 2^{1 - p}$.
Applying Propositions~\ref {prop:comrelerror} and~\ref {prop:relerror}
in the converse direction yields
\begin {equation}
\label {eq:propmulcomrel}
\begin {array}{rl}
\error (\appro x)
&\leq \big( \max (k_{R, 1}, k_{I, 1}) + \max (k_{R, 2}, k_{I, 2}) + 1 \big)
\left( 2 + 2^{\Exp (\appro y) - \Exp (\appro x) + 2} \right)
2^{\Exp (\appro x) - p} \\
\error (\appro y)
&\leq \big( \max (k_{R, 1}, k_{I, 1}) + \max (k_{R, 2}, k_{I, 2}) + 1 \big)
\left( 2 + 2^{\Exp (\appro x) - \Exp (\appro y) + 2} \right)
2^{\Exp (\appro y) - p} \\
\end {array}
\end {equation}


\subsubsection {Norm}
\label {sssec:propnorm}

Let
\[
\appro x = \Norm (\appro {z_1}) = |\appro {z_1}|^2
= \appro {x_1}^2 + \appro {y_1}^2.
\]
Then
\[
\error (\appro x) \leq
| \Norm (\corr {z_1}) - \Norm (\appro {z_1}) |
\leq | \corr {x_1}^2 - \appro {x_1}^2 | + | \corr {y_1}^2 - \appro {y_1}^2 |.
\]
The first term can be bounded by
\begin {align*}
| \corr {x_1}^2 - \appro {x_1}^2 |
& = |\appro {x_1}| \left| 1 + \frac {|\corr {x_1}|}{|\appro {x_1}|} \right|
    |\corr {x_1} - \appro {x_1}| \\
& \leq 2^{\Exp (\appro {x_1})} (2 + \epsilon_{R, 1}^+) k_{R, 1}
2^{\Exp (\appro {x_1}) - p} \\
& \leq k_{R, 1} (2 + \epsilon_{R, 1}^+) 2^{\Exp (\appro {x_1}^2) + 1 - p}
\text { by Proposition~\ref {prop:expmuldiv}} \\
& \leq 2 k_{R, 1} (2 + \epsilon_{R, 1}^+) 2^{\Exp (\appro x) - p}
\text { by the monotonicity of the exponent.}
\end {align*}
The analogous bound for the second error term yields
\begin {equation}
\label {eq:propnorm}
\error (\appro x) \leq
  2 \left(
       k_{R, 1} (2 + \epsilon_{R, 1}^+)
     + k_{I, 1} (2 + \epsilon_{I, 1}^+)
\right)
2^{\Exp (\appro x) - p}
\end {equation}
The values $\epsilon_{X, 1}^+$ may be estimated as explained at the end
of \S\ref {sssec:propmul}.

We also need the relative lower error in the following. This can be obtained
by writing
\[
\appro {x_1}^2 - \corr {x_1}^2
=
\left( 1 - \left| \frac {\corr {x_1}}{\appro {x_1}} \right|^2 \right)
\cdot \appro {x_1}^2
\leq
\big( 1 - (1 - \epsilon_{R, 1}^-)^2 \big) \appro {x_1}^2
=
\big( \epsilon_{R, 1}^- (2 - \epsilon_{R, 1}^-) \big) \appro {x_1}^2.
\]
Adding the corresponding expression for the second term
$\appro {x_1}^2 - \corr {x_1}^2$ yields
\begin {equation}
\label {eq:propnormepsminus}
\frac {\appro x - \corr x}{\appro x}
\leq
\max \big(
   \epsilon_{R, 1}^- (2 - \epsilon_{R, 1}^-),
   \epsilon_{I, 1}^- (2 - \epsilon_{I, 1}^-)
\big)
=: \epsilon^-,
\end {equation}
and under the assumption that $\epsilon^- \geq 0$, inspection of
Definition~\ref {def:relerror} shows that
$\epsilon^- \geq \relerror^- (\appro x)$ since
$\appro x$ and $\corr x$ are positive.

The converse estimation yields
\begin {equation}
\label {eq:propnormepsplus}
\relerror^+ (\appro x)
\leq
\epsilon^+
:=
\frac {\appro x - \corr x}{\appro x}
\leq
\max \big(
   \epsilon_{R, 1}^+ (2 + \epsilon_{R, 1}^+),
   \epsilon_{I, 1}^+ (2 + \epsilon_{I, 1}^+)
\big)
\end {equation}
and $\relerror (\appro x) \leq \epsilon := \max (\epsilon^-, \epsilon^+)$.
Letting
$\epsilon_1 = \max ( \epsilon_{R, 1}^-, \epsilon_{R, 1}^+,
                     \epsilon_{I, 1}^-, \epsilon_{I, 1}^+ )
            = \max ( \epsilon_{R, 1},   \epsilon_{I, 1} )$
and $k_1 = \max ( k_{R, 1}, k_{I, 1})$,
we have
$\epsilon \leq \epsilon_1 (2 + \epsilon_1) \leq 2 k_1 (2 + \epsilon_1) 2^{-p}$
by Proposition~\ref {prop:relerror}.
We obtain an alternative expression for the absolute error as
\begin {equation}
\label {eq:propnormalt}
\error (\appro x) \leq \epsilon \appro x
\leq
2 k_1 (2 + \epsilon_1) 2^{\Exp (\appro x) - p}
\end {equation}


\subsubsection {Division}
\label{sssec:propdiv}

Let
\[
\appro z = \frac {\appro {z_1}}{\appro {z_2}}
= \frac {\appro {z_1} \overline {\appro {z_2}}}{\Norm (\appro {z_2})}.
\]
Then the propagated error may be derived by cumulating the errors obtained
for multiplication in \S\ref {sssec:propmul}, the norm in
\S\ref {sssec:propnorm} and the division by a real in
\S\ref {sssec:proprealdiv}.

Let $\appro {z_3} = \appro {z_1} \overline {\appro {z_2}}
= \appro {x_3} + i \appro {y_3}$,
$d = \Exp (\appro {x_1} \appro {x_2}) - \Exp (\appro {x_3})$
and $d' = \Exp (\appro {y_1} \appro {y_2}) - \Exp (\appro {x_3})$.
Then \eqref {eq:propmulre} applies and yields
$\error (\appro {x_3}) \leq k_{R, 3} 2^{\Exp (\appro {x_3}) - p}$
with
\[
k_{R, 3} = \left( k_{R, 1} (2 + \epsilon_{R, 2}^+)
   + k_{R, 2} (2 + \epsilon_{R, 1}^+) \right) 2^d
   + \left( k_{I, 1} (2 + \epsilon_{I, 2}^+)
   + k_{I, 2} (2 + \epsilon_{I, 1}^+) \right) 2^{d'}.
\]
We then apply \eqref {eq:propnorm} and \eqref {eq:propnormepsminus}
to $\appro {x_4} = \Norm (\appro {z_2})$ to  obtain
$\error (\appro {x_4}) \leq k_4 2^{\Exp (\appro {x_4}) - p}$ with
\[
k_4 =   2 \left(
       k_{R, 2} (2 + \epsilon_{R, 2}^+)
     + k_{I, 2} (2 + \epsilon_{I, 2}^+)
\right)
\]
and
\[
\relerror (\appro {x_4}) = \epsilon_4^-
\leq
\max \big(
   \epsilon_{R, 2}^- (2 - \epsilon_{R, 2}^-),
   \epsilon_{I, 2}^- (2 - \epsilon_{I, 2}^-)
\big).
\]
Now \eqref {eq:proprealdiv} shows that
\begin {equation}
\label {eq:propdivre}
\error (\appro x)
\leq
\frac {2 (k_{R, 3} + k_4)}{1 - \epsilon_4^-} 2^{\Exp (\appro x) - p}.
\end {equation}
As written above, $d$ and $d'$ depend not only on the input and output, but
also on the intermediate value $\appro {x_3}$. But noting that
$
\appro {x_3} = \appro x \Norm (\appro {z_2})
\geq
\appro x 2^{2 \max (\Exp (\appro {x_2}, \Exp (\appro {y_2})) - 2},
$
so that
\begin {align*}
d &\leq
\Exp (\appro {x_1} \appro {x_2}) - \Exp (\appro x)
- 2 \max (\Exp (\appro {x_2}, \Exp (\appro {y_2})) + 2, \\
d' &\leq
\Exp (\appro {y_1} \appro {y_2}) - \Exp (\appro x)
- 2 \max (\Exp (\appro {x_2}, \Exp (\appro {y_2})) + 2,
\end {align*}
yields a bound \eqref {eq:propdivre} that is independent of
intermediate values of the computation.

The error in the imaginary part is computed in the same way as
\begin {equation}
\label {eq:propdivim}
\error (\appro y)
\leq
\frac {2 (k_{I, 3} + k_4)}{1 - \epsilon_4^-} 2^{\Exp (\appro y) - p}.
\end {equation}
with
\begin {align*}
k_{I, 3}
&= \left( k_{R, 1} (2 + \epsilon_{I, 2}^+)
   + k_{I, 2} (2 + \epsilon_{R, 1}^+) \right) 2^\delta
   + \left( k_{I, 1} (2 + \epsilon_{R, 2}^+)
   + k_{R, 2} (2 + \epsilon_{I, 1}^+) \right) 2^{\delta'} \\
\delta &\leq
\Exp (\appro {x_1} \appro {y_2}) - \Exp (\appro y)
- 2 \max (\Exp (\appro {x_2}, \Exp (\appro {y_2})) + 2 \\
\delta' &\leq
\Exp (\appro {x_2} \appro {y_1}) - \Exp (\appro y)
- 2 \max (\Exp (\appro {x_2}, \Exp (\appro {y_2})) + 2
\end {align*}

As for the multiplication, a coarser
bound may be obtained more easily using complex relative errors.
Let $\corr {z_n} = (1 + \theta_n) \appro {z_n}$ with
$\epsilon_n = | \theta_n |$. Then $\corr z = (1 + \theta_n) \appro z$
with
\[
\theta = \frac {1 + \theta_1}{1 + \theta_2} - 1
= (\theta_1 - \theta_2) \sum_{k = 0}^\infty (- \theta_2)^k
\text { and }
\epsilon \leq (|\theta_1| + |\theta_2|) \sum_{k = 0}^\infty |\theta_2|^k.
\]
Using the same notation and assumptions as at the end of
\S\ref {sssec:propmul}, in particular that all higher order error terms
(involving $\epsilon_1^2$, $\epsilon_2^2$, $\epsilon_1 \epsilon_2$
or higher powers) are absorbed by $2^{\Exp (\appro x) - p}$,
we find the exact same error estimate \eqref {eq:propmulcomrel}
also for the case of division.


\subsubsection {Square root}
Let
\[
\appro z = \sqrt {\appro {z_1}}.
\]
Write $\corr {z_1} = (1 + \theta_1) \appro {z_1}$ with
$\epsilon_1 = |\theta_1|$, and assume $\epsilon_1 < 1$.
Then $\corr z = (1 + \theta) \appro z$ with
\[
\theta = \sqrt {1 + \theta_1} - 1
= \frac {1}{2} \theta_1
+ \sum_{n=2}^\infty \frac {(-1)^{n+1} 1 \cdot 3 \cdots (2 n - 3)}{2^n \, n!}
   \theta_1^n
\]
as a Taylor series, and
\[
\epsilon = |\theta|
\leq
\frac {1}{2}  \epsilon_1
+ \sum_{n=2}^\infty \frac {1 \cdot 3 \cdots (2 n - 3)}{2^n \, n!}
\epsilon_1^n
= 1 - \sqrt {1 - \epsilon_1}.
\]
By the mean value theorem, applied to the function $f (x) = \sqrt {1 - x}$,
there is $0 < \xi < \epsilon_1$ with
\begin {equation}
\label {eq:propsqrt}
\epsilon = \frac {1}{2 \sqrt {1 - \xi}} \, \epsilon_1
\leq \frac {1}{2 \sqrt {1 - \epsilon_1}} \, \epsilon_1.
\end {equation}
For instance $\epsilon \leq \epsilon_1$ for $\epsilon_1 \leq \frac {3}{4}$.
We even have $\epsilon \leq \epsilon_1$ for $\epsilon_1 \leq 1$,
since $1 - \sqrt{1-x}$ is bounded by $x$
for $0 < x \leq 1$, which comes from $1 - x \leq \sqrt{1-x}$.


\subsubsection {Logarithm}



\section {Algorithms}

This section describes in detail the algorithms used in \mpc, together with
the error analysis that allows to prove that the results are correct in the
{\mpc} semantics: The input numbers are assumed to be exact, and the output
corresponds to the exact result rounded in the desired direction.


\subsection {\texttt {mpc\_sqrt}}

The following algorithm is due to Friedland \cite{Friedland67,Smith98}.
Let $z = x + i y$.

Let $w = \sqrt { \frac {|x| + \sqrt {x^2 + y^2}}{2}}$ and
$t = \frac {y}{2w}$. Then $(w + it)^2 = |x| + iy$, and with the branch cut on the negative real axis we obtain
\[
\sqrt z = \left\{
\begin {array}{cl}
w + i t & \text {if } x > 0 \\
t + i w & \text {if } x < 0, y > 0 \\
-t - i w & \text {if } x < 0, y < 0
\end {array}
\right.
\]

We compute $w$ rounded down and thus round down all intermediate results.
$\sqrt {x^2 + y^2}$ is computed with an error of \ulp{1}
by a call to \texttt {mpc\_abs}; $|x|$ is added with an error of \ulp{1},
since both terms are positive; division by~$2$ is free of error. So
$w^2$ is computed with a cumulated error of \ulp{2}.
This error of \ulp{2} propagates as is through the real square root:
Since we rounded down the argument, we have $\epsilon_1^- = 0$ in
\eqref {eq:proprealsqrt}; an error of \ulp{1} needs to be added for the
rounding of $w$, so that the total error is \ulp{3}.

$t$ is rounded away. Plugging the error of \ulp{3} for $w$ and \ulp{0} for $y$ into
\eqref {eq:proprealdiv} shows that the propagated error of real division is
\ulp{6}, to which an additional rounding error of \ulp{1} has to be added
for a total error of \ulp{7}.


\subsection {\texttt {mpc\_log}}

Let $z = x + i y$. Then $\log (z) = \frac {1}{2} \log (x^2 + y^2) + i \atantwo (y, x)$. The imaginary part is computed by a call to the corresponding {\mpfr} function.

Let $w = \log (x^2 + y^2)$, rounded down. The error of the complex norm is \ulp{1}. The generic error of the real logarithm is then given by \ulp{$2^{2 - e_w} + 1$}, where $e_w$ is the exponent of $w$. For $e_w \geq 2$, this is bounded by \ulp{2} or 2~digits; otherwise, it is bounded by \ulp{$2^{3 - e_w}$} or $3 - e_w$ digits.

\subsection {\texttt {mpc\_tan}}

Let $z = x + i y$ with $x \neq 0$ and $y \neq 0$.

We compute $\tan z$ as follows:
\begin{align*}
u &\leftarrow \A(\sin z) &\error(\Re(u)) &\leq 1 \Ulp(\Re(u))
&\error(\Im(u)) &\leq 1 \Ulp(\Im(u))
\\
v &\leftarrow \A(\cos z) &\error(\Re(v)) &\leq 1 \Ulp(\Re(v))
&\error(\Im(v)) &\leq 1 \Ulp(\Im(v))
\\
t &\leftarrow \A(u/v) &\error(\Re(t)) &\leq k_R \Ulp(\Re(t))
&\error(\Im(t)) &\leq k_I \Ulp(\Im(t))
\end{align*}
where $w_2 \leftarrow \A(w_1)$ means that the real and imaginary parts of
$w_2$ are respectively the real and imaginary part of $w_1$ rounded away from
zero to the working precision.

We know that $\Re(\frac{a+i b}{c+i d})=\frac{a c +b d}{c^2 + d^2}$ and
$\Im(\frac{a+i b}{c+i d})=\frac{a d -b c}{c^2 + d^2}$, so in the special case
of $\tan z=\frac{\sin x\cosh y+i\cos x\sinh y}{\cos x\cosh y-i\sin x\sinh y}$,
we have $abcd < 0$ which means that there might be a cancellation in the
computation of the real part while it does never happen in the one of the
imaginary part.  Then, using the generic error of the division (see
\ref{sssec:propdiv}), we have
\begin{align*}
\error(\Re(t)) &\leq [1+2^{3+e_1}+2^{3+e_2}+2^6] \Ulp(\Re(t)),
\\
\error(\Im(t)) &\leq [1+2^3+2^3+2^6] \Ulp(\Im(t)),
\end{align*}
where $e_1=\Exp(a c) -\Exp(a c+b d)$ and $e_2=\Exp(b d) -\Exp(a c+b d)$.  The
second inequality shows that $2^7$ is suitable choice for $k_I$. As $|\sinh
y|<\cosh y$ for every nonzero $y$, we have $bd<ac$, thus $e_2\leq e_1$. We
know that $\Exp(\frac{a c+b d}{c^2+d^2})\leq \Exp(a c+b d) -\Exp(c^2+d^2)$,
$\Exp(c^2+d^2)\geq2 \min(\Exp(c), \Exp(d))$, and $\Exp(ac) \leq \Exp(a) +
\Exp(c)$, this gives an upper bound for $e_1$:
\[
e_1 \leq e = \Exp(\Re(u)) +\Exp(\Re(v)) -\Exp(\Re(t))
-2 \min(\Exp(\Re(v)), \Exp(\Im(v))).
\]
and a suitable value for $k_R$:
\begin{equation*}
k_R=\left\{
\begin{array}{l l}
  2^7 & \mbox{if $e < 2$;}
  \\
  2^8 & \mbox{if $e = 2$}
  \\
  2^{5 + e} & \mbox{else.}
\end{array}
\right.
\end{equation*}

\subsection {\texttt {mpc\_pow}}

The main issue for the power function is to be able to recognize when the
real or imaginary part of $x^y$ might be exact, since in that case
Ziv's strategy will loop infinitely.
If both parts of $x^y$ are known to be inexact, then we use
$x^y = \exp(y \log x)$ and Ziv's strategy.
After computing an integer $q$ such that $|y \log x| \leq 2^q$, we first
approximate $y \log x$ with precision $p + q$, and then
$\exp(y \log x)$ with precision $p \geq 4$, all with rounding
to nearest.
Let $\tilde{s} = \round_{p+q}(\log x)$,
we have $\tilde{s} = (\log x) (1 + \theta_1)$
with $\theta_1$ a complex number of norm $\leq 2^{-p-q}$.
Let $\tilde{t} = \round_{p+q}(y \tilde{s})$, then
$\tilde{t} = y \tilde{s} (1 + \theta_2) = (y \log x) (1 + \theta_3)^2$,
where $\theta_2, \theta_3$ are complex numbers of norm $\leq 2^{-p-q}$,
thus $|\tilde{t} - y \log x| \leq 2.5 \cdot 2^{-p}$ for $q \geq -3$.
Now $\tilde{u} = \round_p(\exp(\tilde{t})) =
x^y \exp(2.5 \cdot 2^{-p}) (1 + \theta_4) = x^y (1 + 4 \theta_5)$,
with $\theta_4, \theta_5$ complex numbers of norm $\leq 2^{-p}$.

In the remainder of this section, we determine the cases where at
least one part of $x^y$ is exact, and for that, we assume $x$ to be
different from the trivial cases $0$ and $1$.

\begin {definition}
A {\em dyadic real} is a real number $x$ that is exactly representable
as a floating point number, that is, $x = m \cdot 2^e$ for some $m$, $e \in \Z$.
A {\em dyadic complex} or {\em dyadic}, for short, is a complex number
$x = x_1 + i x_2$ with both $x_1$ and $x_2$ dyadic reals.
\end {definition}

Recall that $\Z [i]$, the ring of Gaussian integers or integers of $\Q (i)$,
is a principal ideal domain with units
$\Z [i]^\ast = \{ \pm 1, \pm i \} = \langle i \rangle$,
in which $2$ is ramified: $(2) = (2 i) = (1 + i)^2$. Let $p_0 = 1 + i$, and
$p_k$ for $k \geq 1$ the remaining primes of $\Z [i]$. Then any element
$x$ of $\Q (i)$ has a unique decomposition as
$x = i^u \prod_{k \geq 0} p_k^{\alpha_k}$ with $u \in \{ 0, 1, 2, 3\}$,
$\alpha_k \in \Z$ and almost all $\alpha_k$ equal to zero.

\begin {prop}
\label {prop:dyadic}
The dyadics are precisely the $p_0$-units of $\Q (i)$, that is,
the numbers $x = i^u \prod_{k \geq 0} p_k^{\alpha_k}$
such that $\alpha_k \geq 0$ for $k \geq 1$.
\end {prop}

\begin {proof}
This follows immediately from the fact that $p_k^{-1}$ for $k \geq 1$ is not
dyadic, while $p_0^{-1} = \frac {1 - i}{2}$ is.
\end {proof}

Gelfond-Schneider's theorem states that if $x$ and $y$ are algebraic and
$y$ is not rational, then $x^y$ is transcendental.
Since all dyadic complex numbers are algebraic, this implies that $x^y$ is
not dyadic whenever $y$ has a non-zero imaginary part.
Unfortunately, this does not rule out the possibility that
either the real or the imaginary part of $x^y$ might still be dyadic,
while the other part is transcendental.
For instance, $i^y$ is real for $y$ purely imaginary, so that
also $x^y$ is real for $x \in \Z [i]^\ast$ and $y$ purely imaginary.

\begin {conj}
\label{conj}
If $\Im y \neq 0$ and $x$ is not a unit of $\Z [i]$, then
the real and the imaginary part of $x^y$ are transcendental.
Or, more weakly, then neither the real nor the imaginary
part of $x^y$ are dyadic reals.
\end {conj}

We then need to examine more closely the case of $y$ a dyadic real,
and we first concentrate on positive $y$.

\begin{lemma}
\label{lemma1}
Let $x$ be a dyadic complex and $m 2^e$ a positive dyadic real
with $m \in \Z_{>0}$, $m$ odd and $e \in \Z$.
Then $x^{m 2^e}$ is a dyadic complex if and only if $x^{2^e}$ is.
\end{lemma}

\begin{proof}
Notice that by Proposition~\ref {prop:dyadic} the set of dyadics forms
a ring, whence any positive integral power of a dyadic is again dyadic.
Thus if $x^{2^e}$ is dyadic, then so is $x^{m 2^e}$.

Conversely, assume that $x^{m 2^e}$ is dyadic. If $e \geq 0$,
then $x^{2^e}$ is dyadic independently of the assumption,
and it remains to consider the case $e < 0$.

Write $x = i^u \prod_{k \geq 0} p_k^{\alpha_k}$
and $z = x^{m 2^e} = i^v \prod_k p_k^{\beta_k}$, so that $x^m = z^{2^{|e|}}$.
The uniqueness of the prime decomposition implies that
$m \alpha_k = 2^{|e|} \beta_k$, and since $m$ is odd, $2^{|e|}$ must
divide $\alpha_k$. Then
$x^{2^e} = i^w \prod_k p_k^{\gamma_k}$ with $w \equiv m^{-1} v \pmod 4$ and
$\gamma_k = \frac {\alpha_k}{2^{|e|}}$.
Now $\alpha_k \geq 0$ for $k \geq 1$ implies $\gamma_k \geq 0$ for $k \geq 1$,
and $x^{2^e}$ is dyadic by Proposition~\ref {prop:dyadic}.
\end{proof}

It remains to decide when $x^{2^e}$ is dyadic for $x$ dyadic. If $e \geq 0$,
this is trivially the case. For $e < 0$, the question boils down to whether
it is possible to take $e$ successive square roots of $x$; as soon as the
process fails, it is clear that $x^{2^e}$ cannot be dyadic.

\begin{lemma}
\label {lm:sqrtrat}
Let $x \in \Q (i)$, and write $x = (a + b i)^2$ with $a$, $b \in \R$.
Then either both of $a$ and $b$ are rational, or none of them is.
\end{lemma}

\begin{proof}
Assume that one of $a$ and $b$ is rational. Then $\Im x = 2 a b \in \Q$
implies that also the other one is rational.
\end{proof}

\begin{lemma}
Let $x$ be dyadic, and write $x = (a + b i)^2$ with $a$, $b \in \R$.
Then either both of $a$ and $b$ are dyadic reals, or none of them is.
\end{lemma}

\begin{proof}
Assume that one of $a$ and $b$ is a dyadic real, that is, a rational with
a power of~$2$ as denominator. Then $a$, $b \in \Q$ by Lemma~\ref {lm:sqrtrat}.
Now, $\Re x = a^2 - b^2$ implies that also the square of the \textit {a priori}
not dyadic coefficient $a$ or $b$, and thus the coefficient itself,
has as denominator a power of~$2$.
\end{proof}


\begin {theorem}
Let $x = m 2^e$ and $y = n 2^f$ be dyadic complex numbers with $m$ and $n$ odd,
and let $z = x^y$. Call the pair $(x, y)$ {\em exceptional} if at least
one of $\Re z$ or $\Im z$ is a dyadic real. Exceptional pairs occur
only in the following cases:
\begin {enumerate}
\item
$y = 0$; then $z = 1$
\item
$x \geq 0$ and $y \neq 0$ are real; then $\Im z = 0$, and the question
whether $\Re z = x^y$ is dyadic involves only real numbers and
can thus be delegated to \mpfr.
\item
$x < 0$ and $y \neq 0$ are real.
\begin {enumerate}
\item
$y \in \Z$; then $\Im z = 0$, and $\Re z = x^y$ is dyadic if and only if
$y > 0$, or $y < 0$ and $-m = 1$.
\item
$y \in \frac {1}{2} \Z \backslash \Z$, that is, $f = -1$;
then $\Re z = 0$, and $\Im z = (-x)^y$ is dyadic if and only if
$e$ is even, $-m$ is a square, and, in case $y < 0$, $-m = 1$.
\item
$y \in \frac {1}{4} \Z \backslash \frac {1}{2} \Z$, that is, $f = -2$;
then $z = \frac {1 + i}{\sqrt 2} (-x)^y$ has both real and imaginary
dyadic parts if and only if
$e \equiv 2 \pmod 4$, $-m$ is a fourth power, and, in case $y < 0$, $-m = 1$.
\end {enumerate}
\item
$y$ not real;
see Conjecture~\ref {conj}
\item
$y > 0$ real, $x$ not real;
see above
\item
$y < 0$ real, $x$ not real;
still to do
\end {enumerate}
\end {theorem}

\begin {proof}
\begin {enumerate}
\item
Clear by definition.
\item
Clear.
\item
The first two subcases $f \geq -1$ follow from the observation that
$x^y = (-1)^y (-x)^y$, where $(-1)^y \in \langle i \rangle$.
For $y > 0$, the number $(-x)^y$ is dyadic if and only if $(-x)^{2^f}$ is,
which leads to the result; for $y < 0$, one furthemore needs that
$(-m)^{-1}$ is dyadic, which for $m$ odd is only possible if $-m = 1$.
The third subcase $f = -2$ is similar, but one needs that $(-x)^y$ is dyadic
up to a factor of $\sqrt 2$.

We proceed to show that for $f \leq -3$, there is no exceptional pair.
Suppose that $(x, y)$ is an exceptional pair; by switching to
$\left( x^{|n|}, \frac {y}{|n|} \right)$, we may assume
without loss of generality that $|n| = 1$. Then $x^y$ is obtained by
taking $|f|$ successive square roots of either $x$ or $\frac {1}{x}$, both
of which are elements of $\Q (i)$. Lemma~\ref {lm:sqrtrat} implies
that both $\Re (x^y)$ and $\Im (x^y)$ are rational.

Write $x^y = \alpha \zeta = \alpha \zeta_r + i \alpha \zeta_i$, where
$\alpha = (-x)^y \in \R$ and $\zeta = \zeta_r + i \zeta_i$ is a primitive root
of unity of order~$2^{|f| + 1}$.
Then $\alpha \zeta_r$, $\alpha \zeta_i \in \Q$ implies $\zeta \in \Q (i, \alpha)$.
Moreover,
$\alpha^2 = \alpha^2 (\zeta_r^2 + \zeta_i^2) =
(\alpha \zeta_r)^2 + (\alpha \zeta_i)^2 \in \Q (i)$, so that $\Q (i, \alpha)$
is an extension of degree at most~$4$ of $\Q$ containing $\Q (\zeta)$
and thus a primitive $16$-th root of unity, which is impossible.
\item
\item
\item
\end {enumerate}
\end {proof}

\paragraph{Sign of zeroes.}
When the output value has a zero real or imaginary part, its sign should be
decided, which is not always possible if we want it to be consistent with the
formula $x^y = \exp(y\log x)$ (in the following, we exclude $0^y$).

Let $x_1$, $x_2$, $y_1$, and $y_2$ real numbers so that $x = x_1 + x_2 i$ and
$y = y_1 + y_2 i$.
Let $\phi \in [-\pi, +\pi]$ the argument of $x = |x| e^{i\phi}$, with the
convention that when $x_1 < 0$ the argument of $x$ is $+\pi$ if $x_2 = +0$ and
$-\pi$ if $x_2 = -0$.
Then
\[
x^y=\exp\left(A(x,y)\right) \left(\cos B(x,y)+\sin B(x,y) i\right)
\] where
\begin {align*}
  A(x,y) & =  y_1\log|x|-y_2\phi,\\
  B(x,y) & =  y_2\log|x|+y_1\phi.
\end {align*}
As $|x^y| = \exp\left(A(x,y)\right)$ is positive, the value of $B(x,y)$
determines the sign of each part of $x^y$.
Note that $A(\overline{x},y) = A(x,\overline{y})$ and $B(\overline{x},
y)=-B(x,\overline{y})$, so $\overline{x}^y = \overline{x^{\overline{y}}}$ and
we can restrict the study below to $x$ with nonnegative imaginary value
(i.e., $x_2 \geq 0$ and $\pi \geq \phi \geq 0$).

To determine the sign of the zero part of $x^y$ when it is pure real or pure
imaginary, special study is needed around points $(x, y)$ where $B(x, y)$ is a
multiple of $\pi/2$.
Let
\begin {equation}
  \label {eqn:Bk}
  B_k(x, y) = y_2 \log|x| +y_1\phi -k\frac{\pi}{2}
\end {equation}
where $k$ is an integer and let $S_k$ the set of points $(x, y)$ where $B_k(x,
y) = 0$.

For any integer $k$, we assume that the surface $S_k$ is orientable and not
reduced to a single point, then each neighborhood of a point $(x_0, y_0)$ of
$S_k$ intersects the region where $B(x, y) > k\pi/2$ and the region where $B(x,
y) < k\pi/2$.
Thus for an even $k$, we can make $(x, y)$ tend continuously to $(x_0, y_0)$
so that $\Re(x^y) > 0$ or we can make it tend to $(x_0, y_0)$ so that
$\Re(x^y) < 0$ (the same applies with $k$ odd and $\Im(x^y)$).
In such cases, the sign of the zero part of $x_0^{y_0}$ is not determined.

However, when $S_k$ intersects an axis, for example when $\Re(x_0) = 0$, we
have to distinguish two cases: $\Re(x_0) = +0$ and $\Re(x_0) = -0$.
Then, if $Q_{x_0}$ (resp. $Q_{y_0}$) denotes the quadrant where $x_0$
(resp. $y_0$) lies, it is possible that the sign of $B_k(x,y)$ remains
constant for $(x,y)$ in the intersection $I$ of a neighborhood of $(x_0, y_0)$
with $Q_{x_0}\times Q_{y_0}$, determining the sign of the zero part of $x^y$.
Let $dB_k(x,y)$ be the derivative of $B_k(x, y)$, we have
\begin {equation}
  \label {eqn:BkDerivative}
  dB_k(x, y)\cdot(\delta_1, \delta_2, \epsilon_1, \epsilon_2) =
  \frac{x_1y_2-x_2y_1}{x_1^2+x_2^2}\delta_1 +
  \frac{x_1y_1+x_2y_2}{x_1^2+x_2^2}\delta_2 +
  \phi\epsilon_1 +
  \log|x| \epsilon_2
\end {equation}
If $dB_k(x_0, y_0)\cdot(\delta_1, \delta_2, \epsilon_1, \epsilon_2)$ is not
zero and if its sign remains constant for all real numbers $\delta_1$,
$\delta_2$, $\epsilon_1$, and $\epsilon_2$ so that $x = x_0 + \delta_1 +
\delta_2i$, $y = y_0 +\epsilon_1 + \epsilon_2i$, and $(x,y)$ is in the given
neighborhood $I$ of $(x_0, y_0)$ defined above, then the sign of $dB_k(x_0,
y_0)\cdot(\delta_1, \delta_2, \epsilon_1, \epsilon_2)$ determines the sign of
the zero part of $x^y$.

In following discussion, we write
$B_k(x_1, x_2, y_1, y_2) := B_k(x_1+x_2i, y_1+y_2i)$,
as a function of four real arguments.
Let $\sigma_1 = -1,+1$ (resp. $\sigma_2$, $\rho_1$, $\rho_2$) denote the sign
of $x_1$ (resp. $x_2$, $y_1$, $y_2$).
\begin {enumerate}
\item Case $B_k(\sigma_1 0, x_2, y_1, y_2)=0$ for $x_2 > 0$.
  Here $\phi = +\frac{\pi}{2}$.
  \begin {enumerate}
  \item if $y_2=\rho_2 0$, then replacing $\phi$ and $y_2$ by their value in
    (\ref{eqn:Bk}), we have $y_1= k$ and (\ref{eqn:BkDerivative}) gives
    \[
    dB_k(\sigma_1 0, x_2, k, \rho_2 0)\cdot(\delta_1, \delta_2, \epsilon_1,
    \epsilon_2) = - \frac{k}{x_2} \delta_1 + 0 \delta_2 + \frac{\pi}{2}
    \epsilon_1 + \log(x_2) \epsilon_2
    \]
    where $\sigma_1 \delta_1 > 0$, and $\rho_2 \epsilon_2 > 0$ so that
    $\delta_1 + (x_2 + \delta_2)i$ (resp. $k+\epsilon_1 + \epsilon_2 i$) is in
    the same quadrant $Q_{x_0}$ (resp.  $Q_{y_0}$) as $x_0=\sigma_1 0 +x_2 i$
    (resp. $y_0=k +\rho_2 0i$).

    When $k \neq 0$, because, in the last expression,
    $\epsilon_1$ would take positive as well as negative values,
    so the sign of $\frac{\pi}{2} \epsilon_1$, and therefore the sign of
    $dB_k(x, y)\cdot (\delta_1, \delta_2, \epsilon_1, \epsilon_2)$ is not
    constant.

    If $k=0$, then $y_1=\rho_1 0$ with $\rho_1 \epsilon_1 > 0$.
    In this case,
    \begin {align*}
      dB_0(\pm 0, x_2, +0, +0)\cdot(\delta_1, \delta_2, \epsilon_1,
      \epsilon_2) &> 0 \;\text{if}\; x_2 \geq 1 \\
      dB_0(\pm 0, x_2, +0, -0)\cdot(\delta_1, \delta_2, \epsilon_1,
      \epsilon_2) &> 0 \;\text{if}\; 1 \geq x_2 > 0 \\
      dB_0(\pm 0, x_2, -0, -0)\cdot(\delta_1, \delta_2, \epsilon_1,
      \epsilon_2) &< 0 \;\text{if}\;  x_2 \geq 1 \\
      dB_0(\pm 0, x_2, -0, +0)\cdot(\delta_1, \delta_2, \epsilon_1,
      \epsilon_2) &< 0 \;\text{if}\; 1 \geq x_2 > 0
    \end {align*}
    and the sign of $dB_k(\sigma_1 0, x_2, \rho_1 0, \rho_2 0)\cdot(\delta_1,
    \delta_2, \epsilon_1, \epsilon_2)$ is not constant in all other
    combinations of $k$, $\sigma_1$, $x_2>0$, $\rho_1$, and $\rho_2$.

  \item if $y_2\neq 0$, then from (\ref{eqn:Bk})
    \[
    x_2 = \exp\left(\frac{k-y_1}{2y_2}\pi\right).
    \]
    But the number in the right hand side of the last equation is
    known to be transcendental unless $k-y_1=0$.
    As $x_2$ is dyadic, we have $y_1= k$ and $x_2=+1$.
    Using (\ref{eqn:BkDerivative}), we write
    \[
    dB_k(\sigma_1 0, +1, k, y_2)\cdot(\delta_1, \delta_2, \epsilon_1,
    \epsilon_2) = - k \delta_1 + y_2 \delta_2 + \frac{\pi}{2} \epsilon_1 + 0
    \epsilon_2
    \]
    where $\sigma_1 \delta_1 > 0$.

    Here, $\delta_2$ can take negative and positive values preventing the sign
    of $dB_k(\sigma_1 0, +1, k, y_2)\cdot (\delta_1, \delta_2, \epsilon_1,
    \epsilon_2)$ from being constant.
  \end {enumerate}

\item Case $B_k(x_1, +0, y_1, y_2)=0$ with $x_1 \neq 0$.
      (Remember we assumed $x_2 \geq 0$.]
  \begin {enumerate}
  \item if $x_1 >0$, then $\phi = +0$.
    From (\ref{eqn:Bk}), we have
    \[
    y_2 \log x_1 = k \frac{\pi}{2}.
    \]
    \begin {enumerate}
    \item if $y_2 = \rho_2 0$, the last equation implies $k = 0$, and from
      (\ref{eqn:BkDerivative}) we have
      \[
      dB_0(x_1, +0, y_1, \rho_2 0)\cdot(\delta_1, \delta_2, \epsilon_1,
      \epsilon_2) = 0 \delta_1 + \frac{y_1}{x_1} \delta_2 + 0 \epsilon_1 +
      \log(x_1) \epsilon_2
      \]
      with $\delta_2 > 0$ and $\rho_2 \epsilon_2 > 0$.

      The sign of the last expression in constant only in the following
      cases,
      \begin {align*}
        dB_0(x_1, +0, y_1, +0)\cdot(\delta_1, \delta_2, \epsilon_1,
        \epsilon_2) &> 0 \;\text{if}\; y_1 \geq 0 \;\text{and}\; x_1 > 1\\
        dB_0(+1, +0, y_1, \pm 0)\cdot(\delta_1, \delta_2, \epsilon_1,
        \epsilon_2) &> 0 \;\text{if}\; y_1 > 0\\
        dB_0(x_1, +0, y_1, -0)\cdot(\delta_1, \delta_2, \epsilon_1,
        \epsilon_2) &> 0 \;\text{if}\; y_1 \geq 0 \;\text{and}\; 1 > x_1 > 0 \\
        dB_0(x_1, +0, y_1, -0)\cdot(\delta_1, \delta_2, \epsilon_1,
        \epsilon_2) &< 0 \;\text{if}\; y_1 \leq 0 \;\text{and}\; x_1 > 1\\
        dB_0(+1, +0, y_1, \pm 0)\cdot(\delta_1, \delta_2, \epsilon_1,
        \epsilon_2) &< 0 \;\text{if}\; y_1 < 0\\
        dB_0(x_1, +0, y_1, +0)\cdot(\delta_1, \delta_2, \epsilon_1,
        \epsilon_2) &< 0 \;\text{if}\; y_1 \leq 0 \;\text{and}\; 1 > x_1 > 0.
      \end {align*}
      Notice that we cannot conclude when $dB_0(x,y)$ is identically zero,
      so the cases $x=+1+0i$, $y=y_1 \pm0i$ cannot be determined by this
      means.

    \item If $y_2 \neq 0$, then $x_1 = \exp\left(\frac{k}{2y_2}\pi\right)$
      is a dyadic number only if $k=0$ and then $x_1=1$.
      We have, using (\ref{eqn:BkDerivative}),
      \[
      dB_0(+1, +0, y_1, y_2)\cdot(\delta_1, \delta_2, \epsilon_1,
      \epsilon_2) = y_2 \delta_1 + y_1 \delta_2 + 0 \epsilon_1 + 0
      \epsilon_2
      \]
      with $\delta_2 > 0$.

      Here $y_2 \delta_1$ can take negative as well as positive values
      preventing the sign of $dB_0(+1, +0, y_1, y_2)\cdot(\delta_1,
      \delta_2, \epsilon_1, \epsilon_2)$ from being constant.
    \end {enumerate}

  \item if $x_1 < 0$, then $\phi = \pi$.
    Using (\ref{eqn:BkDerivative}), we have
    \[
    dB_k(x_1, +0, y_1, y_2)\cdot(\delta_1, \delta_2, \epsilon_1, \epsilon_2)
    = \frac{y_2}{x_1} \delta_1 + \frac{y_1}{x_1} \delta_2 + \pi \epsilon_1 +
    \log(-x_1) \epsilon_2
    \]
    with $\delta_2 > 0$.

    If $y_1 \neq 0$, then $\epsilon_1$ can take negative as well as positive
    values, preventing $dB_k$ from having a constant sign.
    Assume thus $y_1 = 0$.
    As $x_1 \neq 0$, $\delta_1$ can also take negative and positive values,
    and $-y_2 \delta_1$ does not have a constant sign unless $y_2 = 0$.
    But from \ref {eqn:Bk}, we know that $B_k(x, 0)=0$ implies $k = 0$
    since $x_1 = - \exp(\frac{k \pi}{2 y_2})$ is dyadic.

    When $y = 0$, the derivative of $B_k$ is
    \[
    dB_0(x_1, +0, \rho_1 0, \rho_2 0)\cdot(\delta_1, \delta_2, \epsilon_1,
    \epsilon_2) = 0 \delta_1 + 0 \delta_2 + \pi \epsilon_1 + \log(-x_1)
    \epsilon_2
    \]
    with $\delta_2 >0$, $\rho_1 \epsilon_1 >0$ and $\rho_2 \epsilon_2 >0$.

    Then,
    \begin {align*}
      dB_0(x_1, +0, +0, +0)\cdot(\delta_1, \delta_2, \epsilon_1,
      \epsilon_2) &> 0 \;\text{if}\; x_1 \leq -1 \\
      dB_0(x_1, +0, +0, -0)\cdot(\delta_1, \delta_2, \epsilon_1,
      \epsilon_2) &> 0 \;\text{if}\; -1 \leq x_1 < 0 \\
      dB_0(x_1, +0, -0, -0)\cdot(\delta_1, \delta_2, \epsilon_1,
      \epsilon_2) &< 0 \;\text{if}\; x_1 \leq -1\\
      dB_0(x_1, +0, -0, +0)\cdot(\delta_1, \delta_2, \epsilon_1,
      \epsilon_2) &< 0 \;\text{if}\; -1 \leq x_1 < 0
    \end {align*}
    and the sign of $dB_k(x_1, +0, \rho_1 0, \rho_2 0)\cdot(\delta_1,
    \delta_2, \epsilon_1, \epsilon_2)$ is not constant in all other
    combinations of $k$, $x_1<0$, $\rho_1$, and $\rho_2$.
  \end {enumerate}

\item Case $B_k(x_1, x_2, \rho_1 0, y_2)=0$ for $x_2 \geq 0$.
  Here, we have
  \[
  y_2\log|x|-k\frac{\pi}{2} = 0.
  \]
  \begin{enumerate}
  \item If $y_2=0$, then from the last equation $k=0$.
    Using (\ref{eqn:BkDerivative})
    \[
    dB_0(x_1,x_2,\rho_10,\rho_20)\cdot(\delta_1, \delta_2, \epsilon_1, \epsilon_2) =
    0\delta_1 + 0\delta_2 + \phi \epsilon_1 +\log|x| \epsilon_2
    \]
    where $\rho_1 \epsilon_1 > 0$ and $\rho_2 \epsilon_2 > 0$.

    The case $\phi = 0$, that is $x_1 > 0$ and $x_2 = 0$, has already been
    processed above in Case (b)(i).
    Let $\phi > 0$, then $x_2$ is not zero or $x_1 < 0$.
    Using the expression of the derivative given above, we have
    \begin{align*}
      dB_0(x_1, x_2, +0, +0)\cdot(\delta_1, \delta_2, \epsilon_1, \epsilon_2)
      &> 0 \;\text{if}\; |x| \geq 1 \;\text{and}\; \pi \geq \phi > 0\\
      dB_0(x_1, x_2, -0, -0)\cdot(\delta_1, \delta_2, \epsilon_1, \epsilon_2)
      &< 0 \;\text{if}\; |x| \geq 1 \;\text{and}\; \pi \geq \phi > 0\\
      dB_0(x_1, x_2, +0, -0)\cdot(\delta_1, \delta_2, \epsilon_1, \epsilon_2)
      &> 0 \;\text{if}\; 1 \geq |x| > 0 \;\text{and}\; \pi \geq \phi > 0\\
      dB_0(x_1, x_2, -0, +0)\cdot(\delta_1, \delta_2, \epsilon_1, \epsilon_2)
      &< 0 \;\text{if}\; 1 \geq |x| > 0 \;\text{and}\; \pi \geq \phi > 0
    \end{align*}
  \item If $y_2 \neq 0$, from (\ref{eqn:Bk}), we have
    \[
    |x|=\exp\left(\frac{k\pi}{2y_2}\right).
    \]
    The right hand side of the equation is known to be transcendental
    unless $k=0$.
    As the left hand side is dyadic, we have $k=0$, and then $|x|=1$.
    Here, (\ref{eqn:BkDerivative}) gives
    \[
    dB_0(x_1,x_2,\rho_10,y_2)\cdot(\delta_1, \delta_2, \epsilon_1, \epsilon_2)
    = x_1y_2\delta_1 + x_2y_2\delta_2 + \phi \epsilon_1 + 0\epsilon_2
    \]
    with $\rho_1 \epsilon_1 > 0$.

    If $x_2 \neq 0$, then $x_2y_2\delta_2$ can take negative as well as
    positive values and the sign of the derivative is not constant.
    Assume thus $x_2 = 0$.
    Then, $x_1=\pm 1$ because $|x|=1$, and $x_1y_2\delta_1$ can take negative
    and positive values.

    So, the sign of the zero part of $B_k(x_1, x_2, \pm 0, y_2)$ (if any)
    cannot be determined for all integers $k$ and for all dyadic real numbers
    $x_1$, $x_2$, and $y_2 \neq 0$.
  \end{enumerate}

\item If $B_k(x_1, x_2, y_1, \rho_2 0)=0$ for $x_2 \geq 0$.
  The case $y_1=0$ has already been precessed above in Case (c).
  Let $y_1 \neq 0$, then from (\ref{eqn:Bk}) we have
  \[
  \phi = \frac{k}{2y_1}\pi
  \]
  which implies that the argument $\phi$ of $x$ can be written as $r \pi$ for
  some rational number $r$ and, in the same time, $\cos^2 \phi =
  x_1^2/|x|^2$ and $\sin^2 \phi = x_2^2/|x|^2$ are rational.

  The five only possibilities (with $x_2 \geq 0$) are:\footnote{This is wrong:
 consider $\phi = \pi/3$, with $\cos \phi = 1/2$ and $\sin \phi = \sqrt{3}/2$.}
  \begin{enumerate}
  \item $\phi = 0$, but then $x_2=0$.
    This case has been processed above.
  \item $\phi = \frac{\pi}{4}$, then $x_1 = x_2 > 0$.
    From (\ref{eqn:Bk}), we have $y_1 = 2k \neq 0$, and from
    (\ref{eqn:BkDerivative})
    \[
    dB_k(x_1, x_1, 2k, \rho_2 0)\cdot(\delta_1, \delta_2, \epsilon_1,
    \epsilon_2) = -\frac{k}{x_1}\delta_1 + \frac{k}{x_1}\delta_2 +
    \frac{\pi}{4}\epsilon_1 + \log(\sqrt{2}x_1)\epsilon_2
    \]
    with $\rho_2\epsilon_2>0$.

    As $x_1 \neq 0$ and $k \neq 0$, the term $\frac{k}{x_1}\delta_1$, and
    $dB_k(x_1, x_1, 2k, \rho_2 0)\cdot(\delta_1, \delta_2, \epsilon_1,
    \epsilon_2)$ as well, has no constant sign.
  \item $\phi = \frac{\pi}{2}$, then $x_1 = 0$.
    This case has been processed above in Case (a).
  \item $\phi = \frac{3\pi}{4}$, then $x_1 = -x_2$, $x_1 < 0$ and
    (\ref{eqn:Bk}) gives $k \neq 0$ and $y_1 = \frac{2k}{3}$.
    As $y_1$ is a dyadic number, the only compatible values for $k$ are
    multiple of 3.
    Let $n$ be a nonzero integer so that $k = 3n$ and $y_1 = 2n$.
    From (\ref{eqn:BkDerivative}), we have
    \[
    dB_{3n}(-x_2, x_2, 2n, \rho_2 0)\cdot(\delta_1, \delta_2, \epsilon_1,
    \epsilon_2) = -\frac{n}{x_2}\delta_1 - \frac{n}{x_2}\delta_2 +
    \frac{3\pi}{4}\epsilon_1 + \log(\sqrt{2}x_2)\epsilon_2
    \]
    with $\rho_2\epsilon_2 >0$.

    As $n \neq 0$, $\epsilon_1$ can take negative and positive values and
    the term $\frac{3\pi}{4}\epsilon_1$ has not a constant sign.
  \item $\phi = \pi$, then $x_1<0$ and $x_2 = 0$.
    This case has been processed above in Case (b).
  \end{enumerate}
\end {enumerate}

To sum up using the inequalities above and deriving those with negative $x_2$
from them and from the relation $\overline{x}^y =
\overline{x^{\overline{y}}}$, we can give the almost complete list of complex
powers of numbers (for dyadic complex) that have a determined signed zero
part, the only exception being $x=+1 \pm 0i$ raised to zero power which cannot
be treated as we have done here.

\begin{tabular}{r@{ $=$ }lr@{ $=$ }ll}
  $x^{+0 +0i}$ & $1 +0i$,&
  $x^{-0 -0i}$ & $1 -0i$ &
  if $|x|>1$ and $x_2>0$\\
  $x^{-0 +0i}$ & $1 +0i$,&
  $x^{+0 -0i}$ & $1 -0i$ &
  if $|x|>1$ and $x_2<0$\\

  $(x_1 \pm 0i)^{\pm0 +0i}$ & $1 +0i$,&
  $(x_1 \pm 0i)^{\pm0 -0i}$ & $1 -0i$ &
  if $x_1>1$\\

  $(x_1 +0i)^{+0 +0i}$ & $1 +0i$, &
  $(x_1 +0i)^{-0 -0i}$ & $1 -0i$ &
  if $|x_1|>1$\\
  $(x_1 -0i)^{-0 +0i}$ & $1 +0i$, &
  $(x_1 -0i)^{+0 -0i}$ & $1 -0i$ &
  if $|x_1|>1$\\

  $(x_1 +0i)^{y_1 +0i}$ & $x_1^{y_1} +0i$, &
  $(x_1 -0i)^{y_1 -0i}$ & $x_1^{y_1} -0i$ &
  if $x_1>1$ and $y_1>0$\\
  $(x_1 -0i)^{y_1 +0i}$ & $x_1^{y_1} +0i$, &
  $(x_1 +0i)^{y_1 -0i}$ & $x_1^{y_1} -0i$ &
  if $x_1>1$ and $y_1<0$\\

  $(+1 +0i)^{y_1 \pm0i}$ & $1 +0i$, &
  $(+1 -0i)^{y_1 \pm0i}$ & $1 -0i$ &
  if $y_1>0$\\
  $(+1 -0i)^{y_1 \pm0i}$ & $1 +0i$, &
  $(+1 +0i)^{y_1 \pm0i}$ & $1 -0i$ &
  if $y_1<0$\\

  $x^{+0 -0i}$ & $1 +0i$, &
  $x^{-0 +0i}$ & $1 -0i$ &
  if $1>|x|>0$ and $x_2>0$\\
  $x^{-0 -0i}$ & $1 +0i$, &
  $x^{+0 +0i}$ & $1 -0i$ &
  if $1>|x|>0$ and $x_2<0$\\

  $(x_1 \pm0i)^{\pm0 -0i}$ & $1 +0i$, &
  $(x_1 \pm0i)^{\pm0 +0i}$ & $1 -0i$ &
  if $1 > x_1 > 0$ \\

  $(x_1 +0i)^{+0 -0i}$ & $1 +0i$, &
  $(x_1 -0i)^{+0 +0i}$ & $1 -0i$ &
  if $1 > |x_1| > 0$ \\
  $(x_1 -0i)^{-0 -0i}$ & $1 +0i$, &
  $(x_1 +0i)^{-0 +0i}$ & $1 -0i$ &
  if $1 > |x_1| > 0$ \\

  $(x_1 +0i)^{y_1 -0i}$ & $x_1^{y_1} +0i$, &
  $(x_1 -0i)^{y_1 +0i}$ & $x_1^{y_1} -0i$ &
  if $1 > x_1 > 0$ and $y_1 > 0$ \\
  $(x_1 -0i)^{y_1 -0i}$ & $x_1^{y_1} +0i$, &
  $(x_1 +0i)^{y_1 +0i}$ & $x_1^{y_1} -0i$ &
  if $1 > x_1 > 0$ and $y_1 < 0$ \\

  $(\pm 0 +x_2i)^{+0 +0i}$ & $1 +0i$, &
  $(\pm 0 +x_2i)^{-0 -0i}$ & $1 -0i$ &
  if $x_2 \geq 1$ \\
  $(\pm 0 +x_2i)^{+0 -0i}$ & $1 +0i$, &
  $(\pm 0 +x_2i)^{-0 +0i}$ & $1 -0i$ &
  if $1 \geq x_2 > 0$ \\
  $(\pm 0 +x_2i)^{-0 -0i}$ & $1 +0i$, &
  $(\pm 0 +x_2i)^{+0 +0i}$ & $1 -0i$ &
  if $ 0 > x_2 \geq -1$ \\
  $(\pm 0 +x_2i)^{-0 +0i}$ & $1 +0i$, &
  $(\pm 0 +x_2i)^{+0 -0i}$ & $1 -0i$ &
  if $-1 \geq x_2$ \\

  $(-1 +0i)^{+0 \pm0i}$ & $1+0i$, &
  $(-1 +0i)^{-0 \pm0i}$ & $1-0i$ \\
  $(-1 -0i)^{-0 \pm0i}$ & $1+0i$, &
  $(-1 -0i)^{+0 \pm0i}$ & $1-0i$ \\
\end{tabular}

So when $x^y$ is a pure real number, the following pattern is compatible with
the determined cases:

\begin{tabular}{rl}
  $x^y = x_1^{y_1} + \rho_2 0i$ & if $|x| > 1$\\
  $x^y = 1 + \sigma_2 \rho_1 0i$ & if $|x| = 1$\\
  $x^y = x_1^{y_1} - \rho_2 0i$ & if $|x| < 1$\\
  $x^y = x_1^{y_1} + \sigma_2 \rho_1 0i$ & if $y_1 \neq 0$
\end{tabular}

where $\sigma_2$ (resp $\rho_1$, $\rho_2$) is the sign of $x_2$ (resp. $y_1$,
$y_2$) and with the convention $0^0=+1$.

\subsection {\texttt {mpc\_pow\_ui}}
\label {ssec:mpcpowui}

In the case of a positive integer exponent $n$, it may be faster to use
binary exponentiation to compute $x^n$. More generally, let
$n_1, \ldots, n_k$ be an addition chain for $n$, that is, $n_1 = 1$,
$n_k = n$, and for any index $2 \leq r \leq k$, there are indices
$1 \leq s, t \leq r-1$ such that $n_r = n_s + n_t$. Define the corresponding
powers of $x$ as $\corr x_r = x^{n_r}$, so that $\corr x_1 = x$,
$\corr x_k = x^n$ and $\corr x_r = \corr x_s \corr x_t$.
The special case of left-to-right binary exponentiation
satisfies that $n_{r+1} = 2 n_r$ (which occurs $\lfloor \log_2 n \rfloor$
times) or $n_{r+1} = n_r + 1$ (which occurs once less than the number of
$1$ in the binary expansion of $n$, or equivalently, once less than the
Hamming weight of $n$); so $k \leq 2 \lfloor \log_2 n \rfloor + 1$.

Instead of the correct sequence $\corr x_r$, we compute during the algorithm
approximations $\appro x_1 = x = \corr x_1$ and
$\appro x_r = \round (\appro x_s \appro x_t)$
at some precision~$p$.
Let $\theta_r$ be such that $\corr x_r = \appro x_r (1 + \theta_r)$, so that
the relative error of $\appro x_r$ is given by $\epsilon_r = |\theta_r|$.
Write $z_r = \appro x_s \appro x_t$.
Then $\appro x_r = \round (z_r)$ and
$\corr x_r = z_r (1 + \theta_s)(1 + \theta_t)$.

Assuming rounding to nearest, the absolute real error attributable to
rounding $\Re z_r$ to $\Re \appro x_r$ is bounded by
$\frac {1}{2} \cdot 2^{\Exp (\Re z_r) - p}
\leq \frac {1}{2} \cdot 2^{\Exp (\Re \appro x_r) - p}$
by Proposition~\ref {prop:expround}, and the corresponding
relative real error is bounded by $2^{-p}$ according to
Proposition~\ref {prop:relerror}. The same holds for the imaginary part,
and Proposition~\ref {prop:comrelerror} implies that the complex
relative error is also bounded by $2^{-p}$. Otherwise said,
$z_r = \appro x_r (1 + \eta_r)$ with
some complex number $\eta_r$ such that $|\eta_r| \leq 2^{-p}$.
Putting the equations together, we obtain
$1 + \theta_r = (1 + \theta_s)(1 + \theta_t)(1 + \eta_r)$.
We can thus rewrite $1 + \theta_r$ as a product of factors of the
form $1 + \eta$ with $|\eta| \leq 2^{-p}$. Denoting by $u_r$ the
number of such factors, we thus have $u_r = u_s + u_t + 1$, from which
we deduce by induction, using $u_1 = 0 = n_1 - 1$,
that $u_r = n_r - 1$.

So the relative error of $\appro x_k$, compared to $x^n$, is given by
$|\theta_k| \leq (1 + 2^{-p})^{n-1} - 1$.
Using Propositions~\ref {prop:comrelerror}
and~\ref {prop:relerror}, this translates into an absolute error of
\[
\left( 1 + 2^{\Exp (\Im \appro x_k) - \Exp (\Re \appro x_k) + 1} \right)
\left( (1 + 2^{-p})^{n-1} - 1 \right)
2^p \Ulp (\Re \appro x_k)
\]
on the real part and of
\[
\left( 1 + 2^{\Exp (\Re \appro x_k) - \Exp (\Im \appro x_k) + 1} \right)
\left( (1 + 2^{-p})^{n-1} - 1 \right)
2^p \Ulp (\Im \appro x_k)
\]
on the imaginary part of the result.

If we further assume that $(n-1) 2^{-p} \leq 1$, then
$(1 + 2^{-p})^{n-1} - 1 \leq 2 (n - 1) 2^{-p}$,
because $(1+\varepsilon)^m-1 = \exp(m \log(1+\varepsilon)) - 1
\leq \exp(\varepsilon m) - 1 \leq 2 \varepsilon m$ as long as
$\varepsilon m \leq 1$. This gives the simplified bounds
\begin{equation} \label{eq:powui_re}
\left( 2 + 2^{\Exp (\Im \appro x_k) - \Exp (\Re \appro x_k) + 2} \right)
(n-1) \Ulp (\Re \appro x_k)
\end{equation}
on the real part and of
\begin{equation} \label{eq:powui_im}
\left( 2 + 2^{\Exp (\Re \appro x_k) - \Exp (\Im \appro x_k) + 2} \right)
(n-1) \Ulp (\Im \appro x_k)
\end{equation}
on the imaginary part.


\subsection {\texttt {mpc\_pow\_si}}

For the computation of $x^{-n}$ with $n > 0$, the analysis of
\S\ref {ssec:mpcpowui} essentially carries through. We keep the same
notation. Using bounds on the absolute errors of $\Re z_r$ and $\Im z_r$,
we have shown above, using Propositions~\ref {prop:relerror}
and~\ref {prop:comrelerror}, that $z_r = \appro {x_r} (1 + \eta_r)$ with
$|\eta_r| \leq 2^{-p}$ and concluded that
$x^n = \corr {x_k} = \appro {x_k} (1 + \theta_k)$, where
$1 + \theta_k$ is the product of $n-1$ factors of the form
$1 + \eta$  with $|\eta| \leq 2^{-p}$.
By exchanging the roles of $z_r$ and $\appro {x_r}$ and applying
Propositions~\ref {prop:relerror} and~\ref {prop:comrelerror}
analogously, we obtain that $z_r = \frac {\appro {x_r}}{1 + \zeta_r}$ with
$|\zeta_r| \leq 2^{-p}$ and
$x^n = \corr {x_k} = \frac {\appro {x_k}}{1 + \xi_k}$, where
$\xi_k$ is the product of $n-1$ factors of the form
$1 + \zeta$  with $|\zeta| \leq 2^{-p}$.

Let $\corr {x_{k+1}} = \corr {x_k}^{-1} = x^{-n}$ be the desired
result, $z_{k+1} = \appro {x_k}^{-1}$
and $\appro {x_{k+1}} = \round (z_{k+1})$. As shown in \S\ref {ssec:mpcpowui},
rounding of $z_{k+1}$ to the nearest implies that
$z_{k+1} = \appro {x_{k+1}} (1 + \eta_{k+1})$ with
$|\eta_{k+1}| \leq 2^{-p}$.
Then $\corr {x_{k+1}} = \frac {1}{x_k}
= \frac {1 + \xi_k}{\appro {x_k}}
= z_{k+1} (1 + \xi_k)
= \appro {x_{k+1}} (1 + \xi_k)(1 + \eta_{k+1})$,
which contains $n$ factors of the form $1 + \eta$ with
$|\eta| \leq 2^{-p}$.

Thus, assuming that $n 2^{-p} \leq 1$, the error estimates
\eqref {eq:powui_re} and \eqref {eq:powui_im} are valid with $n$
in the place of $n - 1$.

\subsection{\texttt {mpc\_agm1}}

Let
\[
z = \AGM (1, z_1);
\]
for the time being, we assume $\Re (z_1) \geq 0$ and $\Im (z_1) > 0$.

Let $\corr {a_0} = \appro {a_0} = 1$,
$\corr {b_0} = \appro {b_0} = z_1$, and let
$\corr {a_n} = \frac {\corr {a_{n-1}} + \corr {b_{n-1}}}{2}$ and
$\corr {b_n} = \sqrt {\corr {a_{n-1}} \corr {b_{n-1}}}$
be the values of the AGM iteration performed with infinite precision and
$\appro {a_n}, \appro {b_n}$ those performed with $p$-bit precision
and some fixed rounding mode; let $c = \frac {1}{2}$ if this rounding mode
is to nearest, $c = 1$ otherwise.
Let $\eta_n = \relerror (\appro {a_n})$ and
$\epsilon_n = \relerror (\appro {b_n})$.

Let $c_n = \frac {\appro {a_{n-1}} + \appro {b_{n-1}}}{2}$, so that
$\appro {a_n} = \round(c_n)$. By \eqref {eq:propaddrel}, the error of
$c_n$ (relative to the correct value $a_n$) is bounded above by
$\sqrt 2 \, \max (\epsilon_{n-1}, \eta_{n-1})$, division by~$2$ being exact.
After rounding, we obtain
\begin {equation}
\label {eq:agmeta}
\eta_n \leq
\sqrt 2 \, \max (\eta_{n-1}, \epsilon_{n-1})
+ c \left( 1 + \sqrt 2 \, \max (\eta_{n-1}, \epsilon_{n-1}) \right) 2^{1-p}
\end {equation}
by Proposition~\ref {prop:comrelround}.

Let $d_n = \round \left( \appro {a_{n-1}} \appro {b_{n-1}} \right)$.
Then by \eqref {eq:propmulrel} and Proposition~\ref {prop:comrelround},
the error of $d_n$ (relative to $a_{n-1} b_{n-1}$) is bounded above by
\begin {equation}
\label {eq:agmzeta}
\zeta_n =
\eta_{n-1} + \epsilon_{n-1} + \eta_{n-1} \epsilon_{n-1}
+ c (1 + \eta_{n-1} + \epsilon_{n-1} + \eta_{n-1} \epsilon_{n-1}) 2^{1-p}.
\end {equation}
Now $\appro {b_n} = \round (\sqrt {d_n})$, and by \eqref {eq:propsqrt}
and Proposition~\ref {prop:comrelround},
assuming $\zeta_n \leq 1$,
we obtain
\begin {equation}
\label {eq:agmepsilon}
\epsilon_n \leq
\zeta_n + c (1 + \zeta_n) 2^{1-p}.
\end {equation}

Let $r_n = (2^n - 1) d$ for some $d > 2 c$. Then for a sufficiently high
precision~$p$ and not too many steps~$n$ compared to~$p$, we have
$\eta_n, \epsilon_n \leq r_n 2^{1-p}$.
Now we will prove this explicitly in the case $c=1$ and $d=4$,
which is enough to deal with all rounding modes, and
we show
$\eta_n, \epsilon_n \leq r_n 2^{1-p} \leq 2^{n + 3 -p}$
by induction over \eqref {eq:agmeta}, \eqref {eq:agmzeta}
and~\eqref {eq:agmepsilon}. For $n = 0$, we have $\eta_0 = \epsilon_0 = 0$.
Inductively, by \eqref {eq:agmzeta} we obtain
\[
\frac {\zeta_n}{2^{1-p}} \leq
2 r_{n-1} + 1 +
\left( r_{n-1}^2 + 2 r_{n-1} + r_{n-1}^2 2^{1-p} \right) 2^{1-p}
\leq
r_n - 3 + 2^{2n+3-p}
\leq
r_n - 2
\]
as long as $2n+3 \leq p$.\footnote{Indeed,
$r_{n-1}^2 2^{1-p} \leq 2^{2n-2} \cdot 2^4 \cdot 2^{1-p} = 2^{2n+3-p} \leq 1$,
thus $(r_{n-1}^2 + 2 r_{n-1} + r_{n-1}^2 2^{1-p}) 2^{1-p} \leq
(r_{n-1} + 1)^2 2^{1-p} \leq 2^{2n-2} \cdot 2^4 \cdot 2^{1-p} = 2^{2n+3-p}$.}
Then by \eqref {eq:agmepsilon},
\[
\frac {\epsilon_n}{2^{1-p}} \leq
r_n - 1 + r_n 2^{1-p}
\leq
r_n - 1 + 2^{n+3-p}
\leq r_n
\]
under a milder than the previous condition. Finally by \eqref {eq:agmeta},
for $p \geq 3$,
\[
\frac {\eta_n}{2^{1-p}} \leq
\sqrt 2 \, r_{n-1} + 1 + \sqrt {2} \, r_{n-1} 2^{1-p}
\leq \frac{5}{4} \sqrt{2} r_{n-1} + 1 \leq 2 r_{n-1} + d = r_n.
\]

To summarise, we have
\begin {equation}
\label {eq:propagm}
\corr {a_n} = (1 + \theta_1) \appro {a_n}
\text { with }
|\theta_1| \leq 2^{n + 3 - p}
\text { for }
2n+4 \leq p.
\end {equation}

We now use \cite[Prop.~3.3]{Dupont06}, which states that for
$z_1 \neq 0, 1$,
\begin{equation} \label{eq:agmbound}
n \geq B (N, z_1)
  = \max \left( 1, \left\lceil \log_2 |\log_2 |z_1|| \right\rceil \right)
       + \lceil \log_2 (N+4) \rceil + 2
\end{equation}
(where $\log_2 0$ is to be understood as $- \infty$),
we have
\[
a_n = (1 + \theta_2) z
\text { with }
|\theta_2| \leq 2^{-(N+2)}.
\]
So taking $\appro z = \appro {a_n}$ as an approximation for $z$, we obtain
$z = \frac {1 + \theta_1}{1 + \theta_2} \appro z
= (1 + \theta) \appro z$ with
\[
|\theta| \leq \frac {|\theta_1| + |\theta_2|}{|1 - |\theta_2||}
\leq 2 \left( 2^{n+3-p} + 2^{-(N+2)} \right)
\]
(see the computation at the end of \S\ref {sssec:propdiv}).

So after $n = B (N, z_1)$ steps of the AGM iteration at a working precision
of $p = N + n + 5$, we obtain $\appro z = \appro {a_n}$ with a relative error
bounded by $2^{-N}$.

Note that with $p = N + n + 5$, the constraint $2n + 3 \leq p$ gives
$n \leq N+2$. Depending on the value of $z_1$, one might have to take
$N$ larger than the required accuracy to ensure Eq.~(\ref{eq:agmbound}) is
fulfilled.

Using Propositions~\ref {prop:comrelerror} and~\ref {prop:relerror}, this
complex relative error may be translated into an error expressed in $\Ulp$.
With $\appro {z} = \appro x + i \appro y$, let
$k_R = \max (\Exp (\appro y) - \Exp (\appro x) + 1, 0) + 1$, and
$k_I = \max (\Exp (\appro x) - \Exp (\appro y) + 1, 0) + 1$.
Then we have
$\error (\appro x) \leq 2^{k_R - N + p} \Ulp (\appro x)$ and
$\error (\appro y) \leq 2^{k_I - N + p} \Ulp (\appro y)$.

In practice, one should take this additional loss into account:
if rounding fails after the
first computation, nevertheless the values of $k_R$ and $k_I$ will most likely
not change with a larger precision.
Then take $k = \max (k_R, k_I)$,
after $n = B (N + k, z_1)$ steps of the AGM iteration at a working
precision of $p = N + k + n + 5$, one has
$\error (\appro x)
\leq 2^{p - N - (k - k'_R)} \Ulp (\appro x)$
and
$\error (\appro y)
\leq 2^{p - N - (k - k'_I)} \Ulp (\appro y)$,
where $k'_R, k'_I$ denote the new values of $k_R$ and $k_I$.
It then suffices to check a posteriori that $k'_R \leq k$ and
$k'_I \leq k$, then
$\error (\appro x) \leq 2^{p - N} \Ulp (\appro x)$ and
$\error (\appro y) \leq 2^{p - N} \Ulp (\appro y)$.



\bibliographystyle{acm}
\bibliography{algorithms}

\end {document}
