\documentclass {article}

\usepackage[a4paper]{geometry}
\usepackage[utf8]{inputenc}
\usepackage[T1]{fontenc}
\usepackage{amsmath,amssymb}

\newcommand {\mpc}{\texttt {mpc}}
\newcommand {\ulp}[1]{#1~ulp}

\title {MPC: Algorithms and Error Analysis}
\author {The {\mpc} team}
\date {April 10, 2008}


\begin {document}
\maketitle
\tableofcontents


\section {Error analysis}

This section is devoted to the analysis of error propagation: Given a function whose input arguments already have a certain error, what is the error bound on the function output? The output error usually consists of two components: The error propagated from the input, which may be arbitrarily amplified; and an additional small error accounting for the rounding of the output. The results are needed for algorithms that combine several arithmetic operations.


\section {Algorithms}

This section describes in detail the algorithms used in \mpc, together with the error analysis that allows to prove that the results are correct in the {\mpc} semantics: The input numbers are assumed to be exact, and the output corresponds to the exact result rounded in the desired direction.


\subsection {\texttt {mpc\_sqrt}}

Let $z = x + i y$.

Let $w = \sqrt { \frac {|x| + \sqrt {x^2 + y^2}}{2}}$ and
$t = \frac {y}{2w}$. Then $(w + it)^2 = |x| + iy$, and with the branch cut on the negative real axis we obtain
\[
\sqrt z = \left\{
\begin {array}{cl}
w + i t & \text {if } x > 0 \\
t + i w & \text {if } x < 0, y > 0 \\
-t - i w & \text {if } x < 0, y < 0
\end {array}
\right.
\]

$w$ is rounded down. $\sqrt {x^2 + y^2}$ is computed with an error of \ulp{1}; $|x|$ is added with an error of \ulp{1}, since both terms are positive. The generic error of the real square root in the special case that the argument was rounded down is \ulp{1}, so that the total error in computing $w$ is \ulp{3}.

$t$ is rounded up. The generic error of real division, applied to an error of \ulp{3} for $w$ and \ulp{0} for $y$ implies an error of \ulp{7}.
\end {document}
