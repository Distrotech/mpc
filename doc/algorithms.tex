\documentclass {article}

\usepackage[a4paper]{geometry}
\usepackage[utf8]{inputenc}
\usepackage[T1]{fontenc}
\usepackage{amsmath,amssymb}
\usepackage{url}
\usepackage[notref,notcite]{showkeys}

\newcommand {\corr}[1]{\widetilde {#1}}
\newcommand {\appro}[1]{\overline {#1}}
\newcommand {\mpc}{{\tt mpc}}
\newcommand {\mpfr}{{\tt mpfr}}
\newcommand {\ulp}[1]{#1~ulp}
\newcommand {\Ulp}{{\operatorname {ulp}}}
\DeclareMathOperator{\Exp}{\operatorname {Exp}}
\newcommand {\atantwo}{\operatorname {atan2}}
\newcommand{\error}{\operatorname {error}}
\newcommand{\relerror}{\operatorname {relerror}}
\newcommand{\Norm}{\operatorname {N}}
\newcommand {\round}{\operatorname {\circ}}
\DeclareMathOperator{\pinf}{\bigtriangleup}
\DeclareMathOperator{\minf}{\bigtriangledown}
\DeclareMathOperator{\N}{\mathcal N}
\DeclareMathOperator{\A}{\mathcal A}
\newcommand {\Z}{\mathbb Z}
\newcommand {\Q}{\mathbb Q}
\newcommand {\R}{\mathbb R}
\renewcommand {\epsilon}{\varepsilon}
\renewcommand {\theta}{\vartheta}
\renewcommand {\leq}{\leqslant}
\renewcommand {\geq}{\geqslant}

\newtheorem{theorem}{Theorem}
\newtheorem{lemma}[theorem]{Lemma}
\newtheorem{definition}[theorem]{Definition}
\newtheorem{prop}[theorem]{Proposition}
\newtheorem{conj}[theorem]{Conjecture}
\newenvironment{proof}{\noindent{\bf Proof:}}{{\hspace* {\fill}$\blacksquare$}}

\newcommand {\enumi}[1]{(\alph {#1})}
\renewcommand {\labelenumi}{\enumi {enumi}}
\newcommand {\enumii}[1]{(\roman {#1})}
\renewcommand {\labelenumii}{\enumii {enumii}}

\title {MPC: Algorithms and Error Analysis}
\author {Andreas Enge \and Philippe Th\'eveny \and Paul Zimmermann}
\date {June 17, 2009}

\begin {document}
\maketitle
\tableofcontents


\section {Error propagation}

\subsection {Introduction and notation}

This section is devoted to the analysis of error propagation: Given a function
whose input arguments already have a certain error, what is the error bound on
the function output? The output error usually consists of two components: the
error propagated from the input, which may be arbitrarily amplified (or, if
one is lucky, shrunk); and an
additional small error accounting for the rounding of the output. The results
are needed to give a cumulated error analysis for algorithms that combine
several elementary arithmetic operations.


\subsubsection {Ulp calculus}

\begin {definition}
\label {def:exp}
Let $x$ be a real number, which can be written uniquely as
$x = m \cdot 2^e$ with $\frac{1}{2} \le |m| < 1$.
The {\em exponent} of $x$ is
$\Exp(x) = e = \lfloor \log_2 |x| \rfloor + 1$.
The number is {\em representable at precision~$p$} if
$2^p m$ is an integer.
We denote the rounding of $x$ to one of the at most two representable
numbers in the open interval $(x - 2^{-p}, x + 2^{-p})$ by
$\round (x) = \round_p (x)$, with rounding being to nearest, up, down,
towards zero or away from zero if there is a choice.
\end {definition}

\begin {prop}
\label {prop:expmuldiv}
If $x_1$ and $x_2$ are two real numbers, then
\begin {gather*}
\Exp (x_1) + \Exp (x_2) - 1 \leq \Exp (x_1 x_2) \leq \Exp (x_1) + \Exp (x_2),
\\
\Exp (x_1) - \Exp (x_2) \leq \Exp \left( \frac {x_1}{x_2} \right)
\leq \Exp (x_1) - \Exp (x_2) + 1.
\end {gather*}
\end {prop}

\begin {proof}
Write $x_n = m_n 2^{\Exp (x_n)}$ and
$x = x_1 x_2 = m 2^{\Exp x} = m_1 m_2 2^{\Exp (x_1) + \Exp (x_2)}$
with $\frac {1}{2} \leq m_n, m < 1$.
Then $m = m_1 m_2$ if the product is at least $\frac {1}{2}$ and
$m = 2 m_1 m_2$ if the product is less than $\frac {1}{2}$, which
yields the first line of inequalities.
The other inequalities are derived in the same way from
$\frac {1}{2} < \frac {m_1}{m_2} < 2$.
\end {proof}


\begin {prop}
\label {prop:expround}
For any real number $x$,
\[
\Exp (x) \leq \Exp (\round (x)) \leq \Exp (x) + 1,
\]
with equality occurring on the right if and only if
$|x|$ has been rounded up to $|\round (x)| = 2^{\Exp (x)}$.
\end {prop}

\begin {proof}
Letting $x = m 2^{\Exp (x)}$, we have
$\frac {1}{2} \cdot 2^{\Exp (x)} \leq \round (x) \leq 1 \cdot 2^{\Exp (x)}$,
since these two numbers are representable (independently of the precision).
\end {proof}


\begin {definition}
\label {def:ulp}
Let $x$ be a real number which is representable at precision~$p$.
Its associated {\em unit in the last place} is
$\Ulp(x) = \Ulp_p (x) = 2^{\Exp(x) - p}$, so that adding $\Ulp(x)$ to $x$
corresponds to adding $1$ to the integer $2^p m$.
\end {definition}


\subsubsection {Absolute error}

In the remainder of this chapter, all complex numbers are denoted by
the letter $z$ with subscripts and mathematical accents, decomposed in
Cartesian coordinates as $z = x + i y$ with the same diacritics applied
to $x$ and $y$ as to $z$. All representable real numbers are supposed
to have the same precision~$p$. We apply the following error definition
of real numbers separately to the two coordinates of a complex number.

\begin {definition}
\label {def:error}
Given a correct real number $\corr x$ and its approximation $\appro x$,
we define the {\em absolute error} of $\appro x$ as
$\error (\appro x) = | \corr x - \appro x |$.
\end {definition}

Notice that in the following, the absolute error is usually expressed in terms
of $\Ulp$, which is itself a relative measure with respect to the exponent of
the number.

Let $\corr z = f (\corr {z_1}, \ldots) = \corr x + i \corr y$ be the correct
result of a complex function applied to the correct arguments $\corr {z_n}$.
We assume that the $\corr {z_n}$ themselves are not known, but only
approximate input values $\appro {z_n} = \appro {x_n} + i \appro {y_n}$;
for instance, the $\corr {z_n}$ may be the exact results of some formul\ae,
whereas the $\appro {z_n}$ are the outcome of the corresponding computation
and affected by rounding errors. We suppose that error bounds
$\error (\appro {x_n}) \leq k_{R, n} 2^{\Exp (\appro {x_n}) - p}$
and $\error (\appro {y_n}) \leq k_{I, n} 2^{\Exp (\appro {y_n}) - p}$ for
some $k_{R, n}$ and $k_{I, n}$ are known. (This particular notation
becomes more comprehensible when $\appro {x_n}$ and $\appro {y_n}$ are
representable at precision~$p$, since then the units of the error measure
become $\Ulp (\appro {x_n})$ and $\Ulp (\appro {y_n})$, respectively;
however, there is no need to restrict the results of this chapter to
representable numbers.)
Our aim is to determine the propagated error in the output value
$\appro z = \appro x + i \appro y = f (\appro {z_1}, \ldots)$, which is given by
\begin {equation}
\label {eq:properror}
\error (\appro x)
\leq | \Re (f (\corr {z_1}, \ldots)) - \Re (f (\appro {z_1}, \ldots)) |
\end {equation}
and an analogous formula for $\error (\appro y)$. In general,
we are looking for $k_R$ and $k_I$ such that
\[
\error (\appro x) \leq k_R 2^{\Exp (\appro x) - p}
\text { and }
\error (\appro y) \leq k_I 2^{\Exp (\appro y) - p}.
\]
Moreover, we are interested in the cumulated error if additionally
$\appro z$ is rounded coordinatewise at the target precision~$p$
to $\round (\appro z)$. This operation adds an error of
$c_R \Ulp (\round (\appro x))$ to the real and of
$c_I \Ulp (\round (\appro y))$ to the imaginary part, where
$c_X \leq 1$ when $\round$ stands for rounding up, down, to zero or
to infinity, and $c_X \leq \frac {1}{2}$ when $\round$ stands for
rounding to nearest.
Then, via Proposition~\ref {prop:expround},
\[
\error (\round (\appro x)) \leq (k_R + c_R) \Ulp (\appro x)
\text { and }
\error (\round (\appro y)) \leq (k_I + c_I) \Ulp (\appro y).
\]


\subsubsection {Real relative error}

It can sometimes be useful to determine errors not absolutely as differences
(close to~$0$),
but relatively as multiplicative factors (close to~$1$).

\begin {definition}
\label {def:relerror}
Given a correct real number $\corr x$ and its approximation $\appro x$
of the same sign,
we define the {\em lower and upper relative errors} of $\appro x$
as the smallest non-negative real numbers
$\relerror^- (\appro x) = \epsilon^-$ and
$\relerror^+ (\appro x) = \epsilon^+$
such that
\[
1 - \epsilon^- \leq \frac {|\corr x|}{|\appro x|} \leq
1 + \epsilon^+
\]
or, equivalently,
\[
- \epsilon^- \leq \frac {|\corr x| - |\appro x|}{| \appro x| } \leq
\epsilon^+.
\]
The {\em relative error} of $\appro x$ is
\[
\relerror (\appro x) = \epsilon = \max (\epsilon^-, \epsilon^+)
= \frac {\error (\appro x)}{|\appro x|}.
\]
\end {definition}

Notice that $\epsilon^- = 0$ whenever $|\corr x| \geq |\appro x|$
and $\epsilon^+ = 0$ whenever $|\corr x| \leq |\appro x|$, so that
at least one of $\epsilon^+$ and $\epsilon^-$ is zero.
The definition of relative error carries over immediately to complex numbers,
see \S\ref {sssec:comrelerror}.
However, in the following we usually argument separately for the two coordinates,
so we use corresponding $\epsilon$-values with subscript $R$ and $I$ for the
real and imaginary part, respectively.

When an absolute error is expressed in the relative unit $\Ulp$, then
it is easy to switch back and forth between absolute and relative errors.

\begin {prop}
\label {prop:relerror}
Let $\appro x$ be representable at precision $p$.
\begin {enumerate}
\item
If $\error (\appro x) \leq k \Ulp (\appro x)$,
then $\relerror (\appro x) \leq k 2^{1 - p}$.
\item
If $\relerror (\appro x) \leq k 2^{-p}$,
then $\error (\appro x) \leq k \Ulp (\appro x)$.
\end {enumerate}
These assertions remain valid if $\appro x$ is not representable at
precision~$p$ and $\Ulp (\appro x)$ is replaced by $2^{\Exp (\appro x) - p}$.
\end {prop}

\begin {proof}
Concerning the first assertion, we have
$
\relerror (\appro x) = \frac {\error (\appro x)}{|\appro x|}
\leq
\frac {k \Ulp (\appro x)}{|\appro x|}.
$
Plugging in from Definition~\ref {def:ulp} that
$\Ulp (\appro x) = 2^{\Exp (\appro x) - p}$ and
$|\appro x| \geq 2^{\Exp (\appro x) - 1}$ finishes the proof.
The second assertion is proved in the same manner, using
$|\appro x| \leq 2^{\Exp (\appro x)}$.
\end {proof}


\subsubsection {Complex relative error}
\label {sssec:comrelerror}

Some care must be taken when generalising the real relative error to complex
approximations; by keeping the absolute values in
Definition~\ref {def:relerror}, all information is lost.

\begin {definition}
\label {def:comrelerror}
Given a correct complex number $\corr z$ and its non-zero approximation
$\appro z$, let
\[
\theta = \frac {\corr z - \appro z}{\appro z},
\text { or }
\corr z = (1 + \theta) \appro z.
\]
Then the {\em relative error} of $\appro z$ is
\[
\relerror (\appro z) = \epsilon = | \theta |
= \left| \frac {\corr z - \appro z}{\appro z} \right|.
\]
\end {definition}

Notice that this definition coincides with Definition~\ref {def:relerror}
for real numbers of the same sign. The following result gives a coarse
estimate of the relative errors of the real and imaginary parts in terms of
the complex relative error, and vice versa.

\begin {prop}
\label {prop:comrelerror}
Let $\corr z = \corr x + i \corr y$, $\appro z = \appro x + i \appro y$,
$\epsilon = \relerror (\appro z)$,
$\epsilon_R = \relerror (\appro x)$ and
$\epsilon_I = \relerror (\appro y)$,
and assume that the closed circle around $\appro z$ of radius
$\epsilon |\appro z|$ is contained in one quadrant of the complex plane. Then
\begin {align*}
\epsilon_R
&\leq 2^{\max (0.5, \Exp (\appro y) - \Exp (\appro x) + 1.5)} \epsilon \\
\epsilon_I
&\leq 2^{\max (0.5, \Exp (\appro x) - \Exp (\appro y) + 1.5)} \epsilon \\
\epsilon
&\leq \sqrt 2 (\epsilon_R + \epsilon_I)
\end {align*}
\end {prop}

\begin {proof}
By assumption, the correct value $\corr z$ lies in the same quadrant as
$\appro z$, so that $\corr x$ and $\appro x$ resp. $\corr y$ and $\appro y$
have the same sign. Write $\theta = \frac {\corr z - \appro z}{\appro z}
= \theta_R + i \theta_I$. Then
$\corr x - \appro x = \Re (\appro z \theta)
= \appro x \theta_R - \appro y \theta_I$, and
\begin {align*}
\epsilon_R
&= \left| \frac {\corr x - \appro x}{\appro x} \right|
\leq |\theta_R| + \left| \frac {\appro y}{\appro x} \right| |\theta_I|
\leq \max \left( 1, \left| \frac {\appro y}{\appro x} \right| \right)
(|\theta_R| + |\theta_I|) \\
&\leq 2^{\max \left( 0, \Exp (\appro y) - \Exp (\appro x) + 1 \right)}
\cdot \sqrt 2 |\theta|
\end {align*}
by Proposition~\ref {prop:expmuldiv}. The second inequality is proved
in the same way. For the converse direction, write
\[
|\theta_I|
= \left| \Re \left(
\frac {(\corr z - \appro z) \overline z}{\appro x^2 + \appro y^2}
\right) \right|
\leq \left|
\frac {- (\corr x - \appro x) \appro y + (\corr y - \appro y) \appro y}
      {\appro x \appro y} \right|
\leq
  \left| \frac {\corr x - \appro x}{\appro x} \right|
+ \left| \frac {\corr y - \appro y}{\appro y} \right|
\leq \epsilon_R + \epsilon_I,
\]
and similarly for $\theta_R$, which finishes the proof.
\end {proof}



\subsection {Real functions}

In this section, we derive for later use results on error propagation for
functions with real arguments and values. Those already contained in
\cite{MPFRAlgorithms} are simply quoted for the sake of self-containedness.



\subsubsection {Division}
\label {sssec:proprealdiv}

Let
\[
\appro x = \frac {\appro {x_1}}{\appro {x_2}}.
\]
Then
\[
\error (\appro x) = \left|
\frac {\corr {x_1}}{\corr {x_2}} - \frac {\appro {x_1}}{\appro {x_2}} \right|
= \left| \frac {\appro {x_1}}{\appro {x_2}} \right|
\cdot \left|
1 - \left| \frac {\corr {x_1}}{\appro {x_1}} \right|
   \cdot \left| \frac {\appro {x_2}}{\corr {x_2}} \right|
\right|
= | \appro x |
\cdot \left|
1 - \left| \frac {\corr {x_1}}{\appro {x_1}} \right|
   \cdot \left| \frac {\appro {x_2}}{\corr {x_2}} \right|
\right|
\]
Using the notation introduced in Definition~\ref {def:relerror} together
with the obvious subscripts to the $\epsilon$, we obtain the bounds
\[
- \frac {\epsilon_1^+ + \epsilon_2^-}{1 - \epsilon_2^-}
=
1 - \frac {1 + \epsilon_1^+}{1 - \epsilon_2^-}
\leq
1 - \left| \frac {\corr {x_1}}{\appro {x_1}} \right|
   \cdot \left| \frac {\appro {x_2}}{\corr {x_2}} \right|
\leq
1 - \frac {1 - \epsilon_1^-}{1 + \epsilon_2^+}
=
\frac {\epsilon_1^- + \epsilon_2^+}{1 + \epsilon_2^+}
\]
We need to make the assumption that $\epsilon_2^- < 1$, which is reasonable
since otherwise the absolute error on $\appro {x_2}$ would exceed the number
itself. Then
\[
\error (\appro x)
\leq
\max \left(
   \frac {\epsilon_1^+ + \epsilon_2^-}{1 - \epsilon_2^-},
   \frac {\epsilon_1^- + \epsilon_2^+}{1 + \epsilon_2^+}
\right) |\appro x|
\leq
\frac {\epsilon_1 + \epsilon_2}{1 - \epsilon_2^-} 2^{\Exp (\appro x)}
\]
Using the estimation of the relative in terms of the absolute error of
Proposition~\ref {prop:relerror}, this bound can be translated into
\begin {equation}
\label {eq:proprealdiv}
\error (\appro x)
\leq
\frac {2 (k_1 + k_2)}{1 - \epsilon_2^-} 2^{\Exp (\appro x) - p}
\leq
\frac {2 (k_1 + k_2)}{1 - k_2 2^{1 - p}} 2^{\Exp (\appro x) - p}
\end {equation}



\subsection {Complex functions}

\subsubsection {Addition/subtraction}

Using the notation introduced above, we consider
\[
\appro z = \appro {z_1} + \appro {z_2}.
\]
By \eqref {eq:properror}, we obtain
\begin{align*}
\error (\appro x)
& \leq | (\corr {x_1} + \corr {x_2}) - (\appro {x_1} + \appro {x_2})|
\\
& \leq | \corr {x_1} - \appro {x_1} | + | \corr {x_2} - \appro {x_2}|
\\
& \leq k_{R,1} 2^{\Exp (\appro {x_1}) - p}
+ k_{R,2} 2^{\Exp (\appro {x_2}) - p}
\\
& \leq \left( k_{R,1} 2^{d_{R,1}} + k_{R,2} 2^{d_{R,2}} \right)
2^{\Exp (\appro x) - p},
\end{align*}
where $d_{R,n}=\Exp(\appro {x_n})-\Exp(\appro x)$.
Otherwise said, the absolute errors add up, but their relative expression
in terms of $\Ulp$ of the result grows if the result has a smaller
exponent than the operands, that is, if cancellation occurs.

If $\appro {x_1}$ and $\appro {x_2}$, have the same sign, then there
is no cancellation, $d_{R, n} \leq 0$ and
\[
\error (\appro x) \leq (k_{R,1} + k_{R,2} + c_R) 2^{\Exp (\appro x) - p}.
\]

An analogous error bound holds for the imaginary part.

For subtraction, the same bounds are obtained, except that the simpler bound
now holds whenever $\appro {x_1}$ and $\appro {x_2}$ resp.
$\appro {y_1}$ and $\appro {y_2}$ have different signs.


\subsubsection {Multiplication}
\label {sssec:propmul}

Let
\[
\appro z = \appro {z_1} \times \appro {z_2},
\]
so that
\begin {align*}
\appro x & = \appro {x_1} \appro {x_2} - \appro {y_1} \appro {y_2}, \\
\appro y & = \appro {x_1} \appro {y_2} + \appro {x_2} \appro {y_1}.
\end {align*}
Then
\[
\error (\appro x)
\leq | \Re (\corr {z_1} \times \corr {z_2})
- \Re (\appro {z_1} \times \appro {z_2})|
\leq
| \corr {x_1} \corr {x_2} - \appro {x_1} \appro {x_2}|
+ | \corr {y_1} \corr {y_2} - \appro {y_1} \appro {y_2}|.
\]
The first term on the right hand side can be bounded as follows,
where we use the short-hand notation $\epsilon_{R, 1}^+$ for
$\relerror^+ (\appro {x_1})$, and analogously for other relative errors:
\begin{align*}
| \corr {x_1} \corr {x_2} - \appro {x_1} \appro {x_2}|
& \leq
\frac{1}{2} \left(
  |\appro {x_1} - \corr {x_1}| (|\appro {x_2}| + |\corr {x_2}|)
+ |\appro {x_2} - \corr {x_2}| (|\appro {x_1}| + |\corr {x_1}|)
\right)
\\
& \leq \frac {1}{2} \left(
  \epsilon_{R, 1} |\appro {x_1}| |\appro {x_2}|
  \left( 1 + \frac {|\corr {x_2}|}{|\appro {x_2}|} \right)
+ \epsilon_{R, 2} |\appro {x_2}| |\appro {x_1}|
  \left( 1 + \frac {|\corr {x_1}|}{|\appro {x_1}|} \right)
  \right)
\\
& \leq \left(
  k_{R, 1}
  \left( 1 + \frac {|\corr {x_2}|}{|\appro {x_2}|} \right)
+ k_{R, 2}
  \left( 1 + \frac {|\corr {x_1}|}{|\appro {x_1}|} \right)
  \right) |\appro {x_1} \appro {x_2}| \, 2^{-p}
  \text { by Proposition~\ref {prop:relerror}}
\\
& \leq \left(
   k_{R, 1} (2 + \epsilon_{R, 2}^+)
   + k_{R, 2} (2 + \epsilon_{R, 1}^+)
   \right) 2^{\Exp (\appro {x_1} \appro {x_2}) - p}.
\end{align*}
In the same way, we obtain
\[
| \corr {y_1} \corr {y_2} - \appro {y_1} \appro {y_2}|
\leq \left(
   k_{I, 1} (2 + \epsilon_{I, 2}^+)
   + k_{I, 2} (2 + \epsilon_{I, 1}^+)
   \right) 2^{\Exp (\appro {y_1} \appro {y_2}) - p}.
\]

It remains to estimate $\Exp (\appro {x_1} \appro {x_2})$ and
$\Exp (\appro {y_1} \appro {y_2})$ with respect to $\Exp (x)$ to obtain
a bound in terms of $\Ulp (\appro x)$. This becomes problematic when, due
to the subtraction, cancellation occurs. In all generality, let
$d = \Exp (\appro {x_1} \appro {x_2}) - \Exp (\appro x)
\leq \Exp (\appro {x_1}) + \Exp (\appro {x_2}) - \Exp (\appro x)$
by Proposition~\ref {prop:expmuldiv} and
$d' = \Exp( \appro {y_1} \appro {y_2}) - \Exp (\appro x)
\leq \Exp (\appro {y_1}) + \Exp (\appro {y_2}) - \Exp (\appro x)$.
Then
\begin {equation}
\label {eq:propmulre}
\error( \appro x) \leq \left(
   \left( k_{R, 1} (2 + \epsilon_{R, 2}^+)
   + k_{R, 2} (2 + \epsilon_{R, 1}^+) \right) 2^d
   + \left( k_{I, 1} (2 + \epsilon_{I, 2}^+)
   + k_{I, 2} (2 + \epsilon_{I, 1}^+) \right) 2^{d'}
   \right) 2^{\Exp (\appro x) - p}.
\end {equation}
If $\appro {x_1} \appro {x_2}$ and $\appro {y_1} \appro {y_2}$ have different
signs, then there is no cancellation, and, using the monotonicity of the
exponent with respect to the absolute value, we obtain
\[
\Exp (\appro x) = \Exp (\appro {x_1} \appro {x_2} - \appro {y_1} \appro {y_2})
= \Exp (|\appro {x_1} \appro {x_2}| + |\appro {y_1} \appro {y_2}|)
\geq \Exp (|\appro {x_1} \appro {x_2}|), \Exp (|\appro {y_1} \appro {y_2}|),
\]
so that $d$, $d' \leq 0$ and the error bound simplifies as
\[
\error( \appro x) \leq \left(
   k_{R, 1} (2 + \epsilon_{R, 2}^+)
   + k_{R, 2} (2 + \epsilon_{R, 1}^+)
   + k_{I, 1} (2 + \epsilon_{I, 2}^+)
   + k_{I, 2} (2 + \epsilon_{I, 1}^+)
   \right) 2^{\Exp (\appro x) - p}.
\]

The same approach yields the error of the imaginary part. Letting
$\delta = \Exp (\appro {x_1} \appro {y_2}) - \Exp (\appro y)
\leq \Exp( \appro {x_1}) + \Exp (\appro {y_2}) - \Exp (\appro y)$ and
$\delta' = \Exp (\appro {x_2} \appro {y_1}) - \Exp (\appro {y})
\leq \Exp (\appro {x_2}) + \Exp (\appro {y_1}) - \Exp (\appro y)$,
it becomes
\begin {equation}
\label {eq:propmulim}
\error( \appro y) \leq \left(
   \left( k_{R, 1} (2 + \epsilon_{I, 2}^+)
   + k_{I, 2} (2 + \epsilon_{R, 1}^+) \right) 2^{\delta}
   + \left( k_{I, 1} (2 + \epsilon_{R, 2}^+)
   + k_{R, 2} (2 + \epsilon_{I, 1}^+) \right) 2^{\delta'}
   \right) 2^{\Exp (\appro y) - p}.
\end {equation}
If $\appro {x_1} \appro {y_2}$ and $\appro {x_2} \appro {y_1}$ have
the same sign, then $\delta$, $\delta' \leq 0$ and
\[
\error( \appro y) \leq \left(
   k_{R, 1} (2 + \epsilon_{I, 2}^+)
   + k_{I, 2} (2 + \epsilon_{R, 1}^+)
   + k_{I, 1} (2 + \epsilon_{R, 2}^+)
   + k_{R, 2} (2 + \epsilon_{I, 1}^+)
   \right) 2^{\Exp (\appro y) - p}.
\]

The different values $\epsilon_{X, n}^+$ for $X \in \{ R, I \}$ and
$n \in \{ 1, 2 \}$ in the formul{\ae} above may be bounded by
$k_{X, n} 2^{1 - p}$ according to Proposition~\ref {prop:relerror}.
If some $|\appro {x_n}| \geq |\corr {x_n}|$ resp.
$|\appro {y_n}| \geq |\corr {y_n}|$ (for instance, because they have been
computed by rounding away from zero), then the corresponding
$\epsilon_{X, n}^+$ are zero.

A coarser bound may be obtained more easily by considering complex
relative errors. Write $\corr {z_n} = (1 + \theta_n) \appro {z_n}$
with $\epsilon_n = | \theta_n |$. Then $\corr z = (1 + \theta) \appro z$
with $\theta = \theta_1 + \theta_2 + \theta_1 \theta_2$ and
$\epsilon = |\theta| \leq \epsilon_1 + \epsilon_2 + \epsilon_1 \epsilon_2$.
By Proposition~\ref {prop:relerror},
we have $\epsilon_{X, n} \leq k_{X, n} 2^{1-p}$ for $X \in \{ R, I \}$,
and by Proposition~\ref {prop:comrelerror},
$\epsilon_n \leq  (k_{R, n} + k_{I, n}) 2^{1.5 - p}$.
Under normal circumstances, $\epsilon_1 \epsilon_2$ should be negligible,
that is, $\epsilon_1 \epsilon_2
\leq (k_{R, 1} + k_{I, 1}) (k_{R, 2} + k_{I, 2}) 2^{3 - 2 p}
\leq 2^{1.5 - p}$, so that
$\epsilon \leq (k_{R, 1} + k_{I, 1} + k_{R, 2} + k_{I, 2} + 1)
2^{1.5 - p}$.
Applying Propositions~\ref {prop:comrelerror} and~\ref {prop:relerror}
in the converse direction yields, under the assumption that $\corr z$
and $\appro z$ lie in the same quadrant of the complex plane,
\begin {align*}
\error (\appro x)
&\leq (k_{R, 1} + k_{I, 1} + k_{R, 2} + k_{I, 2} + 1)
2^{\max (2, \Exp (\appro y) - \Exp (\appro x) + 3)}
\cdot 2^{\Exp (\appro x) - p} \\
\error (\appro y)
&\leq (k_{R, 1} + k_{I, 1} + k_{R, 2} + k_{I, 2} + 1)
2^{\max (2, \Exp (\appro x) - \Exp (\appro y) + 3)}
\cdot 2^{\Exp (\appro y) - p}
\end {align*}
\subsubsection {Norm}
\label {sssec:propnorm}

Let
\[
\appro x = \Norm (\appro {z_1}) = |\appro {z_1}|^2
= \appro {x_1}^2 + \appro {y_1}^2.
\]
Then
\[
\error (\appro x) \leq
| \Norm (\corr {z_1}) - \Norm (\appro {z_1}) |
\leq | \corr {x_1}^2 - \appro {x_1}^2 | + | \corr {y_1}^2 - \appro {y_1}^2 |.
\]
The first term can be bounded by
\begin {align*}
| \corr {x_1}^2 - \appro {x_1}^2 |
& = |\appro {x_1}| \left| 1 + \frac {|\corr {x_1}|}{|\appro {x_1}|} \right|
    |\corr {x_1} - \appro {x_1}| \\
& \leq 2^{\Exp (\appro {x_1})} (2 + \epsilon_{R, 1}^+) k_{R, 1}
2^{\Exp (\appro {x_1}) - p} \\
& \leq k_{R, 1} (2 + \epsilon_{R, 1}^+) 2^{\Exp (\appro {x_1}^2) + 1 - p}
\text { by Proposition~\ref {prop:expmuldiv}} \\
& \leq 2 k_{R, 1} (2 + \epsilon_{R, 1}^+) 2^{\Exp (\appro x) - p}
\text { by the monotonicity of the exponent.}
\end {align*}
The analogous bound for the second error term yields
\begin {equation}
\label {eq:propnorm}
\error (\appro x) \leq
  2 \left(
       k_{R, 1} (2 + \epsilon_{R, 1}^+)
     + k_{I, 1} (2 + \epsilon_{I, 1}^+)
\right)
2^{\Exp (\appro x) - p}
\end {equation}
The values $\epsilon_{X, 1}^+$ may be estimated as explained at the end
of \S\ref {sssec:propmul}.

We also need the relative lower error in the following. This can be obtained
by writing
\[
\appro {x_1}^2 - \corr {x_1}^2
=
\left( 1 - \left| \frac {\corr {x_1}}{\appro {x_1}} \right|^2 \right)
\cdot \appro {x_1}^2
\leq
\big( 1 - (1 - \epsilon_{R, 1}^-)^2 \big) \appro {x_1}^2
=
\big( \epsilon_{R, 1}^- (2 - \epsilon_{R, 1}^-) \big) \appro {x_1}^2.
\]
Adding the corresponding expression for the second term
$\appro {x_1}^2 - \corr {x_1}^2$ yields
\begin {equation}
\label {eq:propnormepsminus}
\frac {\appro x - \corr x}{\appro x}
\leq
\max \big(
   \epsilon_{R, 1}^- (2 - \epsilon_{R, 1}^-),
   \epsilon_{I, 1}^- (2 - \epsilon_{I, 1}^-)
\big)
=: \epsilon^-,
\end {equation}
and under the assumption that $\epsilon^- \geq 0$, inspection of
Definition~\ref {def:relerror} shows that
$\epsilon^- \geq \relerror^- (\appro x)$ since
$\appro x$ and $\corr x$ are positive.

The converse estimation yields
\begin {equation}
\label {eq:propnormepsplus}
\relerror^+ (\appro x)
\leq
\epsilon^+
:=
\frac {\appro x - \corr x}{\appro x}
\leq
\max \big(
   \epsilon_{R, 1}^+ (2 + \epsilon_{R, 1}^+),
   \epsilon_{I, 1}^+ (2 + \epsilon_{I, 1}^+)
\big)
\end {equation}
and $\relerror (\appro x) \leq \epsilon := \max (\epsilon^-, \epsilon^+)$.
Letting
$\epsilon_1 = \max ( \epsilon_{R, 1}^-, \epsilon_{R, 1}^+,
                     \epsilon_{I, 1}^-, \epsilon_{I, 1}^+ )
            = \max ( \epsilon_{R, 1},   \epsilon_{I, 1} )$
and $k_1 = \max ( k_{R, 1}, k_{I, 1})$,
we have
$\epsilon \leq \epsilon_1 (2 + \epsilon_1) \leq 2 k_1 (2 + \epsilon_1) 2^{-p}$
by Proposition~\ref {prop:relerror}.
We obtain an alternative expression for the absolute error as
\begin {equation}
\label {eq:propnormalt}
\error (\appro x) \leq \epsilon \appro x
\leq
2 k_1 (2 + \epsilon_1) 2^{\Exp (\appro x) - p}
\end {equation}


\subsubsection {Division}
\label{sssec:propdiv}

Let
\[
\appro z = \frac {\appro {z_1}}{\appro {z_2}}
= \frac {\appro {z_1} \overline {\appro {z_2}}}{\Norm (\appro {z_2})}.
\]
Then the propagated error may be derived by cumulating the errors obtained
for multiplication in \S\ref {sssec:propmul}, the norm in
\S\ref {sssec:propnorm} and the division by a real in
\S\ref {sssec:proprealdiv}.
We write
\begin{align*}
\corr a &= \Re (\corr {z_1} \overline {\corr {z_2}}) &
\corr b &= \Im (\corr {z_1} \overline {\corr {z_2}}) &
\corr c &= \Norm (\corr {z_2}) \\
\appro a &= \Re (\appro {z_1} \overline {\appro {z_2}}) &
\appro b &= \Im (\appro {z_1} \overline {\appro {z_2}}) &
\appro c &= \Norm (\appro {z_2})
\end{align*}
Then
\begin {align*}
\error (\appro x)
& \leq \left| \frac {\corr a}{\corr c} - \frac {\appro a}{\appro c} \right| \\
& \leq \frac {1}{\appro c} \left(
\frac {\corr a}{\corr c} |\appro c - \corr c| + |\corr a - \appro a|
\right) \\
& =
\frac {\corr a}{\corr c} \frac {\error (\appro c)}{\appro c}
+ \frac {\error (\appro a)}{\appro c}
\end {align*}
Letting $d = \Exp (\appro {x_1} \appro {x_2}) - \Exp (a)$
and $d' = \Exp (\appro {y_1} \appro {y_2}) - \Exp (a)$, we may apply bound
\eqref {eq:propmulre} to $\appro a$ by replacing $c_R$ by~$0$ since no final
rounding occurs for $\appro a$, and $\Ulp (\appro a)$ by
$2^{\Exp (\appro a) - p}$ since $\appro a$ need not be representable.
Using $\appro x = \round \left( \frac {\appro a}{\appro c} \right)$
and Propositions~\ref {prop:expround} and~\ref {prop:expmuldiv} shows that
$\Exp (\appro a) \leq \Exp (\appro x) + \Exp (\appro c)$; from
$\appro c \geq 2^{\Exp (\appro c) - 1}$ we finally deduce the bound
\[
\frac {\error( \appro a)}{\appro c} \leq \left(
   \left( k_{R, 1} (2 + \epsilon_{R, 2}^+)
   + k_{R, 2} (2 + \epsilon_{R, 1}^+) \right) 2^{d + 1}
   + \left( k_{I, 1} (2 + \epsilon_{I, 2}^+)
   + k_{I, 2} (2 + \epsilon_{I, 1}^+) \right) 2^{d' + 1}
   \right) \Ulp (\appro x).
\]
By \eqref {eq:propnorm} and the previous bound on $\appro c$ we have
\[
\frac {\error (\appro c)}{\appro c} \leq
2 \left (  k_{R,2} (2 + \epsilon^+_{R,2})
  + k_{I,2} (2 + \epsilon^+_{I,2}) \right) 2^{1 - p}.
\]

Let $c_{R,2}^-$ and $c_{I,2}^-$ two positive numbers such that $c_{R,2}^-|x_2|
\leq \left|\corr{x_2}\right|$ and $c_{I,2}^-|y_2| \leq
\left|\corr{y_2}\right|$, then
\[
\left|\frac{A}{C}\right| \leq
\frac{c_{R,1}c_{R,2}+c_{I,1}c_{I,2}}{(c_{R,2}^-)^2+(c_{I,2}^-)^2}
\left|\frac{a}{c}\right|,
\]
and, noticing that $\left|\frac{a}{c}\right|\frac{1}{c}\Ulp(c) \leq 2\Ulp(x)$,
we have
\[
\left|\frac{A}{C}\right|\frac{1}{c}|c-C| \leq
4\frac{c_{R,1}c_{R,2}+c_{I,1}c_{I,2}}{(c_{R,2}^-)^2+(c_{I,2}^-)^2}
\left((1+c_{R,2})k_{R,2}+(1+c_{I,2})k_{I,2}\right)\Ulp(x).
\]
Gathering the relevant inequalities, we show that the error on the real part
is bounded in the following way
\begin{equation*}
  \begin{split}
    \error(x) &\leq [c_R\\
    &\quad+\left((1+c_{R,1})k_{R,2}+(1+c_{R,2})k_{R,1}\right)2^{1+d}
    +\left((1+c_{I,1})k_{I,2}+(1+c_{I,2})k_{I,1}\right)2^{1+d'}\\
    &\quad+4\frac{c_{R,1}c_{R,2}+c_{I,1}c_{I,2}}{(c_{R,2}^-)^2+(c_{I,2}^-)^2}
    \left((1+c_{R,2})k_{R,2}+(1+c_{I,2})k_{I,2}\right)]
    \Ulp(x).
  \end{split}
\end{equation*}

An analog process gives
\begin{equation*}
  \begin{split}
    \error(y) &\leq [c_I\\
    &\quad+\left((1+c_{I,1})k_{R,2}+(1+c_{R,2})k_{I,1}\right)2^{1+\delta}
    +\left((1+c_{R,1})k_{I,2}+(1+c_{I,2})k_{R,1}\right)2^{1+\delta'}\\
    &\quad+4\frac{c_{R,1}c_{R,2}+c_{I,1}c_{I,2}}{(c_{R,2}^-)^2+(c_{I,2}^-)^2}
    \left((1+c_{R,2})k_{R,2}+(1+c_{I,2})k_{I,2}\right)]
    \Ulp(y),
  \end{split}
\end{equation*}
with $\delta=\Exp(y_1x_2)-\Exp(b)$ and $\delta'=\Exp(x_1y_2)-\Exp(b)$.

Note that, when $z_1$ and $z_2$ are the rounded values of $\corr{z_1}$
and $\corr{z_2}$ respectively, with rounding away from zero, then we can
substitute 1 for $k_{R,n}$, $c_{R,n}$, $k_{I,n}$, $c_{I,n}$ (with $n=1,2$),
$\frac{1}{2}$ for $c_{R,2}^-$, $c_{I,2}^-$, giving the much simpler
inequalities:
\begin{align*}
\error(x) &\leq [c_R + 2^{3+d} + 2^{3+d'} + 2^6] \Ulp(x)\\
\error(y) &\leq [c_I + 2^{3+\delta} + 2^{3+\delta'} + 2^6] \Ulp(y).
\end{align*}
If the rounding mode is rounding to nearest, then we can substitute
$\frac{1}{2}$ for $k_{R,n}$, $k_{I,n}$, $c_{R,2}^-$, $c_{I,2}^-$, 2 for
$c_{R,n}$, $c_{I,n}$ and we have:
\begin{align*}
\error(x) &\leq [c_R + 3\times 2^{1+d} +3\times 2^{1+d'} + 192] \Ulp(x)\\
\error(y) &\leq [c_I + 3\times 2^{1+\delta} +3\times 2^{1+\delta'} + 192]
\Ulp(y).
\end{align*}

At last, let us remark that $x_1x_2$ and $y_1y_2$ cannot have the same
sign if $y_1x_2$ and $-x_1y_2$ do have the same sign, thus there is
always a cancellation sometimes in real part, sometimes in imaginary
part of the division.


\subsubsection {Logarithm}



\section {Algorithms}

This section describes in detail the algorithms used in \mpc, together with the error analysis that allows to prove that the results are correct in the {\mpc} semantics: The input numbers are assumed to be exact, and the output corresponds to the exact result rounded in the desired direction.


\subsection {\texttt {mpc\_sqrt}}

The following algorithm is due to Friedland \cite{Friedland67,Smith98}.
Let $z = x + i y$.

Let $w = \sqrt { \frac {|x| + \sqrt {x^2 + y^2}}{2}}$ and
$t = \frac {y}{2w}$. Then $(w + it)^2 = |x| + iy$, and with the branch cut on the negative real axis we obtain
\[
\sqrt z = \left\{
\begin {array}{cl}
w + i t & \text {if } x > 0 \\
t + i w & \text {if } x < 0, y > 0 \\
-t - i w & \text {if } x < 0, y < 0
\end {array}
\right.
\]

$w$ is rounded down. $\sqrt {x^2 + y^2}$ is computed with an error of \ulp{1}; $|x|$ is added with an error of \ulp{1}, since both terms are positive. The generic error of the real square root in the special case that the argument was rounded down is \ulp{1}, so that the total error in computing $w$ is \ulp{3}.

$t$ is rounded up. The generic error of real division, applied to an error of \ulp{3} for $w$ and \ulp{0} for $y$ implies an error of \ulp{7}.


\subsection {\texttt {mpc\_log}}

Let $z = x + i y$. Then $\log (z) = \frac {1}{2} \log (x^2 + y^2) + i \atantwo (y, x)$. The imaginary part is computed by a call to the corresponding {\mpfr} function.

Let $w = \log (x^2 + y^2)$, rounded down. The error of the complex norm is \ulp{1}. The generic error of the real logarithm is then given by \ulp{$2^{2 - e_w} + 1$}, where $e_w$ is the exponent of $w$. For $e_w \geq 2$, this is bounded by \ulp{2} or 2~digits; otherwise, it is bounded by \ulp{$2^{3 - e_w}$} or $3 - e_w$ digits.

\subsection {\texttt {mpc\_tan}}

Let $z = x + i y$ with $x \neq 0$ and $y \neq 0$.

We compute $\tan z$ as follows:
\begin{align*}
u &\leftarrow \A(\sin z) &\error(\Re(u)) &\leq 1 \Ulp(\Re(u))
&\error(\Im(u)) &\leq 1 \Ulp(\Im(u))
\\
v &\leftarrow \A(\cos z) &\error(\Re(v)) &\leq 1 \Ulp(\Re(v))
&\error(\Im(v)) &\leq 1 \Ulp(\Im(v))
\\
t &\leftarrow \A(u/v) &\error(\Re(t)) &\leq k_R \Ulp(\Re(t))
&\error(\Im(t)) &\leq k_I \Ulp(\Im(t))
\end{align*}
where $w_2 \leftarrow \A(w_1)$ means that the real and imaginary parts of
$w_2$ are respectively the real and imaginary part of $w_1$ rounded away from
zero to the working precision.

We know that $\Re(\frac{a+i b}{c+i d})=\frac{a c +b d}{c^2 + d^2}$ and
$\Im(\frac{a+i b}{c+i d})=\frac{a d -b c}{c^2 + d^2}$, so in the special case
of $\tan z=\frac{\sin x\cosh y+i\cos x\sinh y}{\cos x\cosh y-i\sin x\sinh y}$,
we have $abcd < 0$ which means that there might be a cancellation in the
computation of the real part while it does never happen in the one of the
imaginary part.  Then, using the generic error of the division (see
\ref{sssec:propdiv}), we have
\begin{align*}
\error(\Re(t)) &\leq [1+2^{3+e_1}+2^{3+e_2}+2^6] \Ulp(\Re(t)),
\\
\error(\Im(t)) &\leq [1+2^3+2^3+2^6] \Ulp(\Im(t)),
\end{align*}
where $e_1=\Exp(a c) -\Exp(a c+b d)$ and $e_2=\Exp(b d) -\Exp(a c+b d)$.  The
second inequality shows that $2^7$ is suitable choice for $k_I$. As $|\sinh
y|<\cosh y$ for every nonzero $y$, we have $bd<ac$, thus $e_2\leq e_1$. We
know that $\Exp(\frac{a c+b d}{c^2+d^2})\leq \Exp(a c+b d) -\Exp(c^2+d^2)$,
$\Exp(c^2+d^2)\geq2 \min(\Exp(c), \Exp(d))$, and $\Exp(ac) \leq \Exp(a) +
\Exp(c)$, this gives an upper bound for $e_1$:
\[
e_1 \leq e = \Exp(\Re(u)) +\Exp(\Re(v)) -\Exp(\Re(t))
-2 \min(\Exp(\Re(v)), \Exp(\Im(v))).
\]
and a suitable value for $k_R$:
\begin{equation*}
k_R=\left\{
\begin{array}{l l}
  2^7 & \mbox{if $e < 2$;}
  \\
  2^8 & \mbox{if $e = 2$}
  \\
  2^{5 + e} & \mbox{else.}
\end{array}
\right.
\end{equation*}

\subsection {\texttt {mpc\_pow}}

The main issue for the power function is to be able to recognize when the
real or imaginary part of $x^y$ might be exact, since in that case
Ziv's strategy will loop infinitely.
If both parts of $x^y$ are known to be inexact, then we use
$x^y = \exp(y \log x)$ and Ziv's strategy.
After computing an integer $q$ such that $|y \log x| \leq 2^q$, we first
approximate $y \log x$ with precision $p + q$, and then
$\exp(y \log x)$ with precision $p \geq 4$, all with rounding
to nearest.
Let $\tilde{s} = \round_{p+q}(\log x)$,
we have $\tilde{s} = (\log x) (1 + \theta_1)$
with $\theta_1$ a complex number of norm $\leq 2^{-p-q}$.
Let $\tilde{t} = \round_{p+q}(y \tilde{s})$, then
$\tilde{t} = y \tilde{s} (1 + \theta_2) = (y \log x) (1 + \theta_3)^2$,
where $\theta_2, \theta_3$ are complex numbers of norm $\leq 2^{-p-q}$,
thus $|\tilde{t} - y \log x| \leq 2.5 \cdot 2^{-p}$ for $q \geq -3$.
Now $\tilde{u} = \round_p(\exp(\tilde{t})) =
x^y \exp(2.5 \cdot 2^{-p}) (1 + \theta_4) = x^y (1 + 4 \theta_5)$,
with $\theta_4, \theta_5$ complex numbers of norm $\leq 2^{-p}$.

In the remainder of this section, we determine the cases where at
least one part of $x^y$ is exact, and for that, we assume $x$ to be
different from the trivial cases $0$ and $1$.

\begin {definition}
A {\em dyadic real} is a real number $x$ that is exactly representable
as a floating point number, that is, $x = m \cdot 2^e$ for some $m$, $e \in \Z$.
A {\em dyadic complex} or {\em dyadic}, for short, is a complex number
$x = x_1 + i x_2$ with both $x_1$ and $x_2$ dyadic reals.
\end {definition}

Recall that $\Z [i]$, the ring of Gaussian integers or integers of $\Q (i)$,
is a principal ideal domain with units
$\Z [i]^\ast = \{ \pm 1, \pm i \} = \langle i \rangle$,
in which $2$ is ramified: $(2) = (2 i) = (1 + i)^2$. Let $p_0 = 1 + i$, and
$p_k$ for $k \geq 1$ the remaining primes of $\Z [i]$. Then any element
$x$ of $\Q (i)$ has a unique decomposition as
$x = i^u \prod_{k \geq 0} p_k^{\alpha_k}$ with $u \in \{ 0, 1, 2, 3\}$,
$\alpha_k \in \Z$ and almost all $\alpha_k$ equal to zero.

\begin {prop}
\label {prop:dyadic}
The dyadics are precisely the $p_0$-units of $\Q (i)$, that is,
the numbers $x = i^u \prod_{k \geq 0} p_k^{\alpha_k}$
such that $\alpha_k \geq 0$ for $k \geq 1$.
\end {prop}

\begin {proof}
This follows immediately from the fact that $p_k^{-1}$ for $k \geq 1$ is not
dyadic, while $p_0^{-1} = \frac {1 - i}{2}$ is.
\end {proof}

Gelfond-Schneider's theorem states that if $x$ and $y$ are algebraic and
$y$ is not rational, then $x^y$ is transcendental.
Since all dyadic complex numbers are algebraic, this implies that $x^y$ is
not dyadic whenever $y$ has a non-zero imaginary part.
Unfortunately, this does not rule out the possibility that
either the real or the imaginary part of $x^y$ might still be dyadic,
while the other part is transcendental.
For instance, $i^y$ is real for $y$ purely imaginary, so that
also $x^y$ is real for $x \in \Z [i]^\ast$ and $y$ purely imaginary.

\begin {conj}
If $\Im y \neq 0$ and $x$ is not a unit of $\Z [i]$, then
the real and the imaginary part of $x^y$ are transcendental.
Or, more weakly, then neither the real nor the imaginary
part of $x^y$ are dyadic reals.
\end {conj}

We then need to examine more closely the case of $y$ a dyadic real,
and we first concentrate on positive $y$.

\begin{lemma}
\label{lemma1}
Let $x$ be a dyadic complex and $m 2^e$ a positive dyadic real
with $m \in \Z_{>0}$, $m$ odd and $e \in \Z$.
Then $x^{m 2^e}$ is a dyadic complex if and only if $x^{2^e}$ is.
\end{lemma}

\begin{proof}
Notice that by Proposition~\ref {prop:dyadic} the set of dyadics forms
a ring, whence any positive integral power of a dyadic is again dyadic.
Thus if $x^{2^e}$ is dyadic, then so is $x^{m 2^e}$.

Conversely, assume that $x^{m 2^e}$ is dyadic. If $e \geq 0$,
then $x^{2^e}$ is dyadic independently of the assumption,
and it remains to consider the case $e < 0$.

Write $x = i^u \prod_{k \geq 0} p_k^{\alpha_k}$
and $z = x^{m 2^e} = i^v \prod_k p_k^{\beta_k}$, so that $x^m = z^{2^{|e|}}$.
The uniqueness of the prime decomposition implies that
$m \alpha_k = 2^{|e|} \beta_k$, and since $m$ is odd, $2^{|e|}$ must
divide $\alpha_k$. Then
$x^{2^e} = i^w \prod_k p_k^{\gamma_k}$ with $w \equiv m^{-1} v \pmod 4$ and
$\gamma_k = \frac {\alpha_k}{2^{|e|}}$.
Now $\alpha_k \geq 0$ for $k \geq 1$ implies $\gamma_k \geq 0$ for $k \geq 1$,
and $x^{2^e}$ is dyadic by Proposition~\ref {prop:dyadic}.
\end{proof}

It remains to decide when $x^{2^e}$ is dyadic for $x$ dyadic. If $e \geq 0$,
this is trivially the case. For $e < 0$, the question boils down to whether
it is possible to take $e$ successive square roots of $x$; as soon as the
process fails, it is clear that $x^{2^e}$ cannot be dyadic.

\begin{lemma}
\label {lm:sqrtrat}
Let $x \in \Q (i)$, and write $x = (a + b i)^2$ with $a$, $b \in \R$.
Then either both of $a$ and $b$ are rational, or none of them is.
\end{lemma}

\begin{proof}
Assume that one of $a$ and $b$ is rational. Then $\Im x = 2 a b \in \Q$
implies that also the other one is rational.
\end{proof}

\begin{lemma}
Let $x$ be dyadic, and write $x = (a + b i)^2$ with $a$, $b \in \R$.
Then either both of $a$ and $b$ are dyadic reals, or none of them is.
\end{lemma}

\begin{proof}
Assume that one of $a$ and $b$ is a dyadic real, that is, a rational with
a power of~$2$ as denominator. Then $a$, $b \in \Q$ by Lemma~\ref {lm:sqrtrat}.
Now, $\Re x = a^2 - b^2$ implies that also the square of the \textit {a priori}
not dyadic coefficient $a$ or $b$, and thus the coefficient itself,
has as denominator a power of~$2$.
\end{proof}


\begin {theorem}
Let $x = m 2^e$ and $y = n 2^f$ be dyadic complex numbers with $m$ and $n$ odd,
and let $z = x^y$. Call the pair $(x, y)$ {\em exceptional} if at least
one of $\Re z$ or $\Im z$ is a dyadic real. Exceptional pairs occur
only in the following cases:
\begin {enumerate}
\item
$y = 0$; then $z = 1$
\item
$x \geq 0$ and $y \neq 0$ are real; then $\Im z = 0$, and the question
whether $\Re z = x^y$ is dyadic involves only real numbers and
can thus be delegated to \mpfr.
\item
$x < 0$ and $y \neq 0$ are real.
\begin {enumerate}
\item
$y \in \Z$; then $\Im z = 0$, and $\Re z = x^y$ is dyadic if and only if
$y > 0$, or $y < 0$ and $-m = 1$.
\item
$y \in \frac {1}{2} \Z \backslash \Z$, that is, $f = -1$;
then $\Re z = 0$, and $\Im z = (-x)^y$ is dyadic if and only if
$e$ is even, $-m$ is a square, and, in case $y < 0$, $-m = 1$.
\item
$y \in \frac {1}{4} \Z \backslash \frac {1}{2} \Z$, that is, $f = -2$;
then $z = \frac {1 + i}{\sqrt 2} (-x)^y$ has both real and imaginary
dyadic parts if and only if
$e \equiv 2 \pmod 4$, $-m$ is a fourth power, and, in case $y < 0$, $-m = 1$.
\end {enumerate}
\item
$y$ not real;
see Conjecture
\item
$y > 0$ real, $x$ not real;
see above
\item
$y < 0$ real, $x$ not real;
still to do
\end {enumerate}
\end {theorem}

\begin {proof}
\begin {enumerate}
\item
Clear by definition.
\item
Clear.
\item
The first two subcases $f \geq -1$ follow from the observation that
$x^y = (-1)^y (-x)^y$, where $(-1)^y \in \langle i \rangle$.
For $y > 0$, the number $(-x)^y$ is dyadic if and only if $(-x)^{2^f}$ is,
which leads to the result; for $y < 0$, one furthemore needs that
$(-m)^{-1}$ is dyadic, which for $m$ odd is only possible if $-m = 1$.
The third subcase $f = -2$ is similar, but one needs that $(-x)^y$ is dyadic
up to a factor of $\sqrt 2$.

We proceed to show that for $f \leq -3$, there is no exceptional pair.
Suppose that $(x, y)$ is an exceptional pair; by switching to
$\left( x^{|n|}, \frac {y}{|n|} \right)$, we may assume
without loss of generality that $|n| = 1$. Then $x^y$ is obtained by
taking $|f|$ successive square roots of either $x$ or $\frac {1}{x}$, both
of which are elements of $\Q (i)$. Lemma~\ref {lm:sqrtrat} implies
that both $\Re (x^y)$ and $\Im (x^y)$ are rational.

Write $x^y = \alpha \zeta = \alpha \zeta_r + i \alpha \zeta_i$, where
$\alpha = (-x)^y \in \R$ and $\zeta = \zeta_r + i \zeta_i$ is a primitive root
of unity of order~$2^{|f| + 1}$.
Then $\alpha \zeta_r$, $\alpha \zeta_i \in \Q$ implies $\zeta \in \Q (i, \alpha)$.
Moreover,
$\alpha^2 = \alpha^2 (\zeta_r^2 + \zeta_i^2) =
(\alpha \zeta_r)^2 + (\alpha \zeta_i)^2 \in \Q (i)$, so that $\Q (i, \alpha)$
is an extension of degree at most~$4$ of $\Q$ containing $\Q (\zeta)$
and thus a primitive $16$-th root of unity, which is impossible.
\item
\item
\item
\end {enumerate}
\end {proof}

A relevant reference is \cite{BrDiJeLeMeMuReStTo09}, especially Section 4.5
which discusses complex floating-point numbers, and gives error bounds for
multiplication, division and square root.

\bibliographystyle{acm}
\bibliography{algorithms}

\end {document}
