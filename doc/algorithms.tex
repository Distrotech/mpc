\documentclass {article}

\usepackage[a4paper]{geometry}
\usepackage[utf8]{inputenc}
\usepackage[T1]{fontenc}
\usepackage{amsmath,amssymb}

\newcommand {\mpc}{\texttt {mpc}}
\newcommand {\mpfr}{\texttt {mpfr}}
\newcommand {\ulp}[1]{#1~ulp}
\newcommand {\atantwo}{\operatorname {atan2}}
\DeclareMathOperator{\error}{error}
\newcommand {\Ulp}{{\rm ulp}}
\newcommand {\Exp}{{\rm \textsc exp}}

\title {MPC: Algorithms and Error Analysis}
\author {Andreas Enge \and Philippe Th\'eveny}
\date {April 11, 2008}

\begin {document}
\maketitle
\tableofcontents


\section {Error analysis}

This section is devoted to the analysis of error propagation: Given a function whose input arguments already have a certain error, what is the error bound on the function output? The output error usually consists of two components: The error propagated from the input, which may be arbitrarily amplified; and an additional small error accounting for the rounding of the output. The results are needed for algorithms that combine several arithmetic operations.

\subsection {Generic error for addition/substraction}

Let $z_1 = \circ(\tilde{z_1})$ and $z_2 = \circ(\tilde{z_2})$, with
$\tilde{z_n} = \tilde{x_n} + i \tilde{y_n}$ and $z_n = x_n + i y_n$,
the two operand of the addition rounded at the working precision, the error on
real and imaginary parts are $\error(x_n) \leq k_{R,n} \Ulp(x_n)$, and 
$\error(y_n) \leq k_{I,n} \Ulp(y_n)$.

\begin{eqnarray}
z&=&\circ(\tilde{z_1}+\tilde{z_2})\\\nonumber
\end{eqnarray}
\begin{eqnarray}\nonumber
\error(x)&=&|x-\Re(\tilde{z_1}-\tilde{z_2})|\\\nonumber
&\leq&|x-\Re(z_1-z_2)|+|\Re(z_1+z_2)-\Re(\tilde{z_1}+\tilde{z_2})|\\\nonumber
&\leq&c_R \Ulp(x) + |x_1 - \tilde{x_1}| + |x_2 - \tilde{x_2}|\\\nonumber
&\leq&c_R \Ulp(x) + k_{R,1} \Ulp(x_1) + k_{R,2} \Ulp(x_2)\\\nonumber
&\leq&\left(c_R + 2^{d_{R,1}} k_{R,1} + 2^{d_{R,2}} k_{R,2} \right) \Ulp(x)
\nonumber
\end{eqnarray}
where $d_{R,n}=\Exp(x_n)-\Exp(x)$ and $c_R=\frac{1}{2}$ if the real part is
rounded to nearest, else $c_R=1$. If $x_1x_2 \geq 0$, then we have a simpler
expression
\[
\error(x) \leq (c_R+k_{R,1}+k_{R,2}) \Ulp(x)
\]
In the same way, we can show that the generic error of the imaginary part is
\[
\error (y) \leq \left(c_I + 2^{d_{I,1}} k_{I,1} + 2^{d_{I,2}} k_{I,2} \right) \Ulp(y)
\]
where $d_{I,n}=\Exp(y_n)-\Exp(y)$ and $c_I=\frac{1}{2}$ if the imaginary part
is rounded to nearest, else $c_I=1$. If $y_1y_2 \geq 0$, then we have the simpler expression
\[
\error(y) \leq (c_I+k_{I,1}+k_{I,2}) \Ulp(y)
\]

\section {Algorithms}

This section describes in detail the algorithms used in \mpc, together with the error analysis that allows to prove that the results are correct in the {\mpc} semantics: The input numbers are assumed to be exact, and the output corresponds to the exact result rounded in the desired direction.


\subsection {\texttt {mpc\_sqrt}}

Let $z = x + i y$.

Let $w = \sqrt { \frac {|x| + \sqrt {x^2 + y^2}}{2}}$ and
$t = \frac {y}{2w}$. Then $(w + it)^2 = |x| + iy$, and with the branch cut on the negative real axis we obtain
\[
\sqrt z = \left\{
\begin {array}{cl}
w + i t & \text {if } x > 0 \\
t + i w & \text {if } x < 0, y > 0 \\
-t - i w & \text {if } x < 0, y < 0
\end {array}
\right.
\]

$w$ is rounded down. $\sqrt {x^2 + y^2}$ is computed with an error of \ulp{1}; $|x|$ is added with an error of \ulp{1}, since both terms are positive. The generic error of the real square root in the special case that the argument was rounded down is \ulp{1}, so that the total error in computing $w$ is \ulp{3}.

$t$ is rounded up. The generic error of real division, applied to an error of \ulp{3} for $w$ and \ulp{0} for $y$ implies an error of \ulp{7}.


\subsection {\texttt {mpc\_log}}

Let $z = x + i y$. Then $\log (z) = \frac {1}{2} \log (x^2 + y^2) + i \atantwo (y, x)$. The imaginary part is computed by a call to the corresponding {\mpfr} function.

Let $w = \log (x^2 + y^2)$, rounded down. The error of the complex norm is \ulp{1}. The generic error of the real logarithm is then given by \ulp{$2^{2 - e_w} + 1$}, where $e_w$ is the exponent of $w$. For $e_w \geq 2$, this is bounded by \ulp{2} or 2~digits; otherwise, it is bounded by \ulp{$2^{3 - e_w}$} or $3 - e_w$ digits.

\end {document}
